\documentclass{beamer}
\usepackage{csc}
\title{Лекция 11. Библиотека Boost.}

\date{
   \textbf{ИТМО JB}\\
   16 ноября 2021 \\
   Санкт-Петербург
}

\newcommand{\specialcell}[2][c]{%
  \begin{tabular}[#1]{@{}c@{}}#2\end{tabular}}

\begin{document}
\begin{frame} 
  \titlepage
\end{frame}

\begin{frame}[fragile]{Boost}
    \begin{itemize}
        \item Коллекция библиотек, расширяющих функциональность \langcpp.
        \item Свободно распространяются по лицензии Boost Software License вместе с исходным кодом и документацией на 
        \url{www.boost.org}.
        \item Лицензия позволяет использовать boost в коммерческих проектах.
        \item Библиотеки из boost являются кандидатами на включение\\ в стандарт \langcpp.
        \item Некоторые библиотеки boost были включены в стандарты\\ \langcpp 2014/17 года (\texttt{std::variant}, \texttt{std::optional}, \ldots).
        \item При включении библиотеки в boost она проходит\\ несколько этапов рецензирования.
        \item Библиотеки boost позволяют обеспечить переносимость.
        \item В текущей версии в boost более сотни библиотек.
    \end{itemize}
\end{frame}

\begin{frame}[fragile]{Boost}
    \begin{itemize}
        \item Существенно замедляют компиляцию ваших проектов.
        \item Достаточно часто встречаются баги (в отличие от STL).
        \item Существенно увеличивают размер библиотеки/исполняемого файла.
    \end{itemize}
\end{frame}

\begin{frame}[fragile]{Категории библиотек Boost}
\medskip
\fontsize{10pt}{0.8em}\selectfont
\begin{minipage}{.5\textwidth}

\begin{itemize}
\item String and text processing
\item Containers, 
\item Iterators
\item Algorithms
\item Function objects and higher-order programming
\item Generic Programming
\item Template Metaprogramming
\item Concurrent Programming
\item Math and numerics
\item Correctness and testing
\item Data structures
\item Domain Specific
\item System
\end{itemize}
\end{minipage}%
\begin{minipage}{.45\textwidth}
\begin{itemize}
\item Input/Output
\item Memory
\item Image processing
\item Inter-language support
\item Language Features Emulation
\item Parsing
\item Patterns and Idioms
\item Programming Interfaces
\item State Machines
\item Broken compiler workarounds
\item Preprocessor Metaprogramming
\end{itemize}
\end{minipage}%
\end{frame}

\begin{frame}[fragile]{{\tt program\_options}}
\small
\begin{lstlisting}
#include <boost/program_options.hpp> 

namespace po = boost::program_options;

int main() {
    po::options_description desc("Allowed options");
    desc.add_options()
        ("help", "produce help message")
        ("comp", po::value<int>(), "compression level");

    po::variables_map vm;
    po::store(po::parse_command_line(ac, av, desc), vm);
    po::notify(vm);    

    ...
}
\end{lstlisting}
\end{frame}

\begin{frame}[fragile]{{\tt program\_options}}
\small
\begin{lstlisting}
int main() {
    ...  

    if (vm.count("help")) {
        cout << desc << endl;
        return 1;
    }
    
    if (vm.count("comp")) {
        const auto comp_level = vm["comp"].as<int>();
    } else {
        cout << "Compression level was not set." << endl;
    }

    ...
}
\end{lstlisting}
\end{frame}

\begin{frame}[fragile]{{\tt any}}
\lstset{basicstyle=\fontsize{7pt}{.8em}\ttfamily,  
  keywordstyle=\fontsize{7pt}{.8em}\color{MOOCBlue}\bfseries, 
  commentstyle=\fontsize{7pt}{.8em}\ttfamily\color{MOOCGreen}}
\small
\begin{lstlisting}
#include <any>

using namespace std;

int main() {
    any a = 1;
    // i:1
    cout << format("{}:{}", a.type().name(), any_cast<int>(a)) << endl;
    a = 3.14;
    // d:3.14
    cout << format("{}:{}", a.type().name(), any_cast<double>(a)) << endl;
    try {
        a = 1;
        cout << any_cast<float>(a) << '\n';
    }
    catch (const bad_any_cast & e) {
        cout << e.what() << endl;
    }
}
\end{lstlisting}
\end{frame}

\begin{frame}[fragile]{{\tt any}}
\lstset{basicstyle=\fontsize{7pt}{.8em}\ttfamily,  
  keywordstyle=\fontsize{7pt}{.8em}\color{MOOCBlue}\bfseries, 
  commentstyle=\fontsize{7pt}{.8em}\ttfamily\color{MOOCGreen}}
\small
\begin{lstlisting}
#include <any>
#include <list>

using namespace std;
using many = list<any>;

void append_int(many & values, int value) {
    any to_append = value;
    values.push_back(to_append);
}
void append_string(many & values, const string & value) {
    values.push_back(value);
}
void append_char_ptr(many & values, const char * value) {
    values.push_back(value);
}
void append_any(many & values, const any & value) {
    values.push_back(value);
}
...
\end{lstlisting}
\end{frame}

\begin{frame}[fragile]{{\tt function}}
\lstset{basicstyle=\fontsize{8pt}{.8em}\ttfamily,  
  keywordstyle=\fontsize{8pt}{.8em}\color{MOOCBlue}\bfseries, 
  commentstyle=\fontsize{8pt}{.8em}\ttfamily\color{MOOCGreen}}
\small
\begin{lstlisting}
std::function<float(int x, int y)> f;
struct int_div { 
    float operator()(int x, int y) const { return ((float)x)/y; }; 
};
f = int_div();
std::cout << f(5, 3) << std::endl;

std::function<void(int values[], int n, int& sum, float& avg)> sum_avg;
void do_sum_avg(int values[], int n, int& sum, float& avg) { ... }
sum_avg = &do_sum_avg;

struct X { int foo(int); };
std::function<int(X*, int)> f;
f = &X::foo;
X x;
f(&x, 5);
\end{lstlisting}
\end{frame}

\begin{frame}[fragile]{{\tt bind}}
\begin{lstlisting}
struct image;

struct animation {
    void advance(int ms);
    bool inactive() const;
    void render(image & target) const;
};

std::vector<animation> anims;

template<class C, class P>
void erase_if(C & c, P pred) {
    c.erase(
        std::remove_if(c.begin(), c.end(), pred),
        c.end());
}
\end{lstlisting}
\end{frame}

\begin{frame}[fragile]{{\tt bind}}
\begin{lstlisting}
void update(int ms) {
    std::for_each(
        anims.begin(),
        anims.end(), 
        std::bind(&animation::advance, _1, ms));
    erase_if(anims, std::mem_fn(&animation::inactive));
}

void render(image & target) {
    std::for_each(
        anims.begin(),
        anims.end(), 
        std::bind(&animation::render, _1,
                  std::ref(target)));
}
\end{lstlisting}
\end{frame}

\begin{frame}[fragile]{{\tt bind}}
\begin{lstlisting}

void update(int ms) {
    for (auto && anim : anims) {
        anim.advance(ms);
    }
    erase_if(
        anims,
        [](auto & anim) { return anim.inactive(); });
}

void render(image & target) {
    for (auto && anim : anims) {
        anim.render(target);
    }
}
\end{lstlisting}
\end{frame}

\begin{frame}[fragile]{{\tt bind} and {\tt function}}
\lstset{basicstyle=\fontsize{7pt}{.8em}\ttfamily,  
  keywordstyle=\fontsize{7pt}{.8em}\color{MOOCBlue}\bfseries, 
  commentstyle=\fontsize{7pt}{.8em}\ttfamily\color{MOOCGreen}}
\small
\begin{lstlisting}
struct button 
{
    std::function<void()> onClick;
};

struct player
{
    void play();
    void stop();
};

button playButton, stopButton;
player thePlayer;

void connect()
{
    // equal to lambda wrapper
    playButton.onClick = std::bind(&player::play, &thePlayer);
    stopButton.onClick = std::bind(&player::stop, &thePlayer);
}

\end{lstlisting}
\end{frame}

\begin{frame}[fragile]{{\tt signal}}
\lstset{basicstyle=\fontsize{7pt}{.8em}\ttfamily,  
  keywordstyle=\fontsize{7pt}{.8em}\color{MOOCBlue}\bfseries, 
  commentstyle=\fontsize{7pt}{.8em}\ttfamily\color{MOOCGreen}}
\small
\begin{lstlisting}
#include <boost/signal.hpp>

using namespace std;

struct A {
    boost::signal<void()> SigA;
    boost::signal<void(int)> SigB;
};

struct B {
    void PrintFoo() { cout << "Foo" << endl; }
    void PrintBar(int i) { cout << "Bar: " << i << endl; }
};

int main() {
    A a; B b;

    // connect
    a.SigA.connect(bind(&B::PrintFoo, &b));
    a.SigB.connect(bind(&B::PrintBar, &b, _1));

    // post message
    a.SigA();
    a.SigB(4);
}
\end{lstlisting}
\end{frame}

\begin{frame}[fragile]{{\tt signal}}
\lstset{basicstyle=\fontsize{7pt}{.8em}\ttfamily,  
  keywordstyle=\fontsize{7pt}{.8em}\color{MOOCBlue}\bfseries, 
  commentstyle=\fontsize{7pt}{.8em}\ttfamily\color{MOOCGreen}}
\small
\begin{lstlisting}
#include <boost/signal.hpp>

using namespace std;

struct A {
    boost::signal<void()> SigA;
    boost::signal<void(int)> SigB;
};

struct B {
    void PrintFoo() { cout << "Foo" << endl; }
    void PrintBar(int i) { cout << "Bar: " << i << endl; }
};

int main() {
    A a; B b;

    // connect
    a.SigA.connect([&] { b.PrintFoo(); });
    a.SigB.connect([&](auto i) { b.PrintBar(i); });

    // post message
    a.SigA();
    a.SigB(4);
}
\end{lstlisting}
\end{frame}

\begin{frame}[fragile]{{\tt lexical\_cast}}
\lstset{basicstyle=\fontsize{7pt}{.8em}\ttfamily,  
  keywordstyle=\fontsize{7pt}{.8em}\color{MOOCBlue}\bfseries, 
  commentstyle=\fontsize{7pt}{.8em}\ttfamily\color{MOOCGreen}}
\small
\begin{lstlisting}
#include <boost/lexical_cast.hpp>

int main(int, char ** argv) {
    std::vector<short> args;
    while (*++argv) {
        try {
            args.push_back(boost::lexical_cast<short>(*argv));
        }
        catch(const boost::bad_lexical_cast &) {
            args.push_back(0);
        }
    }
} 

void log_message(const std::string &);

void log_errno(int yoko) {
    log_message(
        fmt::format("Error {}: {}",
        boost::lexical_cast<std::string>(yoko),
        strerror(yoko)));
}
\end{lstlisting}
\end{frame}


\begin{frame}[fragile]{{\tt optional}}
\lstset{basicstyle=\fontsize{7pt}{.8em}\ttfamily,  
  keywordstyle=\fontsize{7pt}{.8em}\color{MOOCBlue}\bfseries, 
  commentstyle=\fontsize{7pt}{.8em}\ttfamily\color{MOOCGreen}}
\small
\begin{lstlisting}
optional<char> get_async_input() {
    if ( !queue.empty() )
        return optional<char>(queue.top());
    else return optional<char>(); // uninitialized
}

void receive_async_message() {
    optional<char> rcv ;
    // The safe boolean conversion from 'rcv' is used here.
    while ( (rcv = get_async_input()) && !timeout() )
        output(*rcv);
}
\end{lstlisting}
\end{frame}

\begin{frame}[fragile]{String Algo}
\lstset{basicstyle=\fontsize{7pt}{.8em}\ttfamily,  
  keywordstyle=\fontsize{7pt}{.8em}\color{MOOCBlue}\bfseries, 
  commentstyle=\fontsize{7pt}{.8em}\ttfamily\color{MOOCGreen}}
\small
\begin{lstlisting}
#include <boost/algorithm/string.hpp>
using namespace std;
using namespace boost;

string str1(" hello world! ");
to_upper(str1);  // str1 == " HELLO WORLD! "
trim(str1);      // str1 == "HELLO WORLD!"

string str2= to_lower_copy(
  ireplace_first_copy(str1,"hello","goodbye")); // \verb!str2 == "goodbye world!"!

string str3("hello abc-*-ABC-*-aBc goodbye");
using find_vector_type = vector<iterator_range<string::iterator>>;

find_vector_type FindVec; // \verb!#1: Search for separators!
ifind_all(FindVec, str3, "abc"); // \verb!{ [abc],[ABC],[aBc] }!

using split_vector_type = vector<string>;

split_vector_type SplitVec; // \verb!#2: Search for tokens!
split(SplitVec, str3, is_any_of("-*-"), token_compress_on); 
// \verb!{ "hello abc","ABC","aBc goodbye" }!
\end{lstlisting}
\end{frame}

\begin{frame}[fragile]{variant}
\lstset{basicstyle=\fontsize{7pt}{.8em}\ttfamily,  
  keywordstyle=\fontsize{7pt}{.8em}\color{MOOCBlue}\bfseries, 
  commentstyle=\fontsize{7pt}{.8em}\ttfamily\color{MOOCGreen}}
\small
\begin{lstlisting}
#include "boost/variant.hpp"
#include <iostream>

struct my_visitor : public boost::static_visitor<int> {
    int operator()(int i) const {
        return i;
    }
    int operator()(const std::string & str) const {
        return str.length();
    }
};

int main() {
    boost::variant<int, std::string> u("hello world");
    std::cout << u; // output: hello world

    int result = boost::apply_visitor(my_visitor(), u);
    std::cout << result; 
    // output: 11 (i.e., length of "hello world")
}
\end{lstlisting}
\end{frame}

\begin{frame}[fragile]{filesystem}
\lstset{basicstyle=\fontsize{7pt}{.8em}\ttfamily,  
  keywordstyle=\fontsize{7pt}{.8em}\color{MOOCBlue}\bfseries, 
  commentstyle=\fontsize{7pt}{.8em}\ttfamily\color{MOOCGreen}}
\begin{lstlisting}

// using namespace std::filesystem or boost::filesystem

int main(int argc, char **argv)
{
    path p(argv[1]);

    if (exists(p)) 
    {
        if (is_regular_file(p))
            cout << "size is " << file_size(p);
        else if (is_directory(p))
            cout << "is a directory";
        else
            cout << "exists, but is neither a regular file nor a directory";
    }
    else
        cout << p << "does not exist\n";
}
\end{lstlisting}
\end{frame}

\end{document}

 
