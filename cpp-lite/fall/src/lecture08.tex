\documentclass{beamer}
\usepackage{csc}
\title{Лекция 8. Perfect-forwarding. Контейнеры.}

\date{
   \textbf{ИТМО JB}\\
   26 октября 2021 \\
   Санкт-Петербург
}

\begin{document}
\begin{frame} 
  \titlepage
\end{frame}

\begin{frame}[fragile]{Преобразование ссылок в шаблонах}
    ``Склейка'' ссылок:
    \begin{itemize}
    \item \texttt{T\& \&\ \ \ $\to$ T\&}
    \item \texttt{T\& \&\&\ \  $\to$ T\&}
    \item \texttt{T\&\& \&\ \  $\to$ T\&}
    \item \texttt{T\&\& \&\&\  $\to$ T\&\&}
    \end{itemize}

\begin{block}{Универсальная или пересылающая ссылка}
\begin{lstlisting}
template<typename T>
void foo(T && t) {}
\end{lstlisting}
\begin{itemize}
    \item Если вызвать {\tt foo} от lvalue типа {\tt A}, то {\tt T = A\&}.
    \item Если вызвать {\tt foo} от rvalue типа {\tt A}, то {\tt T = A}.
\end{itemize}
\end{block}
\end{frame}

\begin{frame}[fragile]{Как работает {\tt std::move}?}
Определение \texttt{std::move}:
\begin{lstlisting}
template<class T> 
typename remove_reference<T>::type&&
   move(T&& a) 
{
   typedef typename remove_reference<T>::type&& RvalRef;
   return static_cast<RvalRef>(a);
}  
    \end{lstlisting}


\begin{block}{Важно}
    \texttt{std::move} не выполняет никаких действий\\
    времени выполнения.
\end{block}
\end{frame}

\begin{frame}[fragile]{{\tt std::move} для lvalue}
Вызываем \texttt{std::move} для lvalue объекта.
\begin{lstlisting}
X x;
x = std::move(x);
\end{lstlisting}
Тип \texttt{T} выводится как \texttt{X\&}.
    \begin{lstlisting}
typename remove_reference<X&>::type&&
   move(X& && a) 
{
   using RvalRef = typename remove_reference<X&>::type&&;
   return static_cast<RvalRef>(a);
} 
    \end{lstlisting}
После склейки ссылок получаем:
    \begin{lstlisting}
X&& move(X& a) 
{
   return static_cast<X&&>(a);
} 
    \end{lstlisting}
\end{frame}

\begin{frame}[fragile]{{\tt std::move} для rvalue}
Вызываем \texttt{std::move} для временного объекта.
\begin{lstlisting}
X x = std::move(X());
\end{lstlisting}
Тип \texttt{T} выводится как \texttt{X}.
    \begin{lstlisting}
typename remove_reference<X>::type&&
   move(X&& a) 
{
   using RvalRef = typename remove_reference<X>::type&&;
   return static_cast<RvalRef>(a);
} 
    \end{lstlisting}
После склейки ссылок получаем:
    \begin{lstlisting}
X&& move(X&& a) 
{
   return static_cast<X&&>(a);
} 
    \end{lstlisting}

\end{frame}


\begin{frame}[fragile]{Perfect forwarding}
    \begin{lstlisting}
// для lvalue
template<typename T, typename Arg> 
auto make_unique(Arg & arg) { 
    return unique_ptr<T>(new T(arg));
} 

// для rvalue
template<typename T, typename Arg> 
auto make_unique(Arg && arg) { 
    return unique_ptr<T>(new T(std::move(arg)));
} 
    \end{lstlisting}

\texttt{std::forward} позволяет записать это одной функцией.
    \begin{lstlisting}
template<typename T, typename Arg> 
auto make_unique(Arg&& arg) { 
    return unique_ptr<T>(new T(std::forward<Arg>(arg)));
} 
    \end{lstlisting}
\end{frame}

\begin{frame}[fragile]{Как работает \texttt{std::forward}?}
Определение \texttt{std::forward}:
    \begin{lstlisting}
template<class S>
S&& forward(typename remove_reference<S>::type& a) 
{
    return static_cast<S&&>(a);
} 
    \end{lstlisting}

\begin{block}{Важно}
    \texttt{std::forward} не выполняет никаких действий\\
    времени выполнения.
\end{block}
\end{frame}

\begin{frame}[fragile]{\texttt{std::forward} для lvalue}
\begin{lstlisting}
X x;
auto p = make_unique<A>(x); 	// Arg = X\&
\end{lstlisting}

\begin{lstlisting}
unique_ptr<A> make_unique(X& && arg) { 
  return unique_ptr<A>(new A(std::forward<X&>(arg)));
} 

X& && forward(remove_reference<X&>::type& a) {
  return static_cast<X& &&>(a);
} 
\end{lstlisting}
\end{frame}

\begin{frame}[fragile]{\texttt{std::forward} для lvalue}
После склейки ссылок:
\begin{lstlisting}
unique_ptr<A> make_unique(X& arg) { 
  return unique_ptr<A>(new A(std::forward<X&>(arg)));
} 

X& forward(X& a) {
  return static_cast<X&>(a);
} 
\end{lstlisting}
\end{frame}

\begin{frame}[fragile]{\texttt{std::forward} для rvalue}
\begin{lstlisting}
auto p = make_unique<A>(X());	// Arg = X
\end{lstlisting}

\begin{lstlisting}
unique_ptr<A> make_unique(X&& arg) { 
  return unique_ptr<A>(new A(std::forward<X>(arg)));
} 

X&& forward(remove_reference<X>::type& a) {
  return static_cast<X&&>(a);
} 
\end{lstlisting}
\end{frame}

\begin{frame}[fragile]{\texttt{std::forward} для rvalue}
После склейки ссылок:
\begin{lstlisting}
unique_ptr<A> make_unique(X&& arg) { 
  return unique_ptr<A>(new A(std::forward<X>(arg)));
} 

X&& forward(X& a) {
  return static_cast<X&&>(a);
} 
\end{lstlisting}

\end{frame}

\begin{frame}[fragile]{Variadic templates $+$ perfect forwarding}
Можно применить \texttt{std::forward} для списка параметров.
\begin{lstlisting}
template<typename T, typename ...Args> 
std::unique_ptr<T> make_unique(Args&&... args) {
    return std::unique_ptr<T>(
                new T(std::forward<Args>(args)...));
}
\end{lstlisting}
Теперь \texttt{make\_unique} работает для произвольного числа аргументов.
\begin{lstlisting}
auto p = make_unique<Array<string>>(10, string("Hello"));
\end{lstlisting}
\end{frame}

\begin{frame}{Общие сведения о контейнерах}
\small
    Контейнеры библиотеки STL можно разделить на две категории: 
    \begin{itemize}
        \item последовательные, 
        \item ассоциативные.
    \end{itemize}
    \begin{block}{Общие вложенные типы}
    \begin{itemize} \tt
        \item Container::value\_type
        \item Container::iterator, Container::const\_iterator
    \end{itemize}
    \end{block}\vspace{-3mm}
    \begin{block}{Общие методы контейнеров}
    \begin{itemize}
    \item \texttt{size, max\_size, empty, clear}.
    
    \item \texttt{begin, end, сbegin, сend}.

    \item Операторы сравнения: \texttt{==, !=, >, >=, <, <=}.
    \end{itemize}
\end{block}
\end{frame}

\begin{frame}[fragile]{Шаблон {\tt array}}
    Класс-обёртка над статическим массивом.

\begin{itemize}
    \item \texttt{operator[], at},
    \item {\tt back, front}.
    \item \texttt{fill},
    \item \texttt{data}.
\end{itemize}
Позволяет работать с массивом как с контейнером.

\begin{lstlisting}
std::array<std::string, 3> a = {"One", "Two", "Three"};
std::cout << a.size() << std::endl;
std::cout << a[1] << std::endl;

// ошибка времени выполнения
std::cout << a.at(3) << std::endl; 
\end{lstlisting}
\end{frame}

\begin{frame}{Общие методы остальных последовательных контейнеров}
\begin{itemize}
    \item Конструктор от двух итераторов.  
    \item Конструктор от \texttt{count} и \texttt{defVal}.
    \item Конструктор от \texttt{std::initializer\_list<T>}.
    \item Методы {\tt back, front}.
    \item Методы \texttt{push\_back, emplace\_back}    
    \item Методы {\tt assign}.
    \item Методы {\tt insert}. 
    \item Meтоды \texttt{emplace}.
    \item Методы \texttt{erase} от одного и двух итераторов. 
\end{itemize}
\end{frame}

\begin{frame}[fragile]{Шаблон {\tt vector}}
    Динамический массив с автоматическим изменением размера\\ при
    добавлении элементов.

\begin{itemize}
    \item \texttt{operator[], at},
    \item \texttt{resize},
    \item \texttt{capacity, reserve, shrink\_to\_fit},
    \item \texttt{pop\_back},
    \item \texttt{data}.
\end{itemize}

\begin{lstlisting}
std::vector<std::string> v = {"One", "Two"};
v.reserve(100);
v.push_back("Three");
v.emplace_back("Four");
// Позволяет работать со старым кодом.
legacy_function(v.data(), v.size());
std::cout << v[2] << std::endl;
\end{lstlisting}
\end{frame}
           
\begin{frame}[fragile]{Шаблон {\tt deque}}
Контейнер c возможностью быстрой вставки и удаления элементов на
обоих концах за $O(1)$. Реализован как список указателей на массивы
фиксированного размера.
\begin{itemize}
    \item {\tt operator[], at},
    \item {\tt resize},
    \item {\tt push\_front, emplace\_front}
    \item \texttt{pop\_back, pop\_front},
    \item \texttt{shrink\_to\_fit}.
\end{itemize}
\begin{lstlisting}
std::deque<std::string> d = {"One", "Two"};
d.emplace_back("Three");
d.emplace_front("Zero");
std::cout << d[1] << std::endl;
\end{lstlisting}

\end{frame}

\begin{frame}[fragile]{Шаблон {\tt list}}
Двусвязный список. В любом месте контейнера 
вставка и удаление производятся за $O(1)$.
\begin{itemize}
    \item {\tt push\_front, emplace\_front},
    \item \texttt{pop\_back, pop\_front},
    \item \texttt{splice},
    \item {\tt merge, remove, remove\_if, reverse, \\sort, unique}.
\end{itemize}
\begin{lstlisting}
std::list<std::string> l = {"One", "Two"};
l.emplace_back("Three");
l.emplace_front("Zero");
std::cout << l.front() << std::endl;
\end{lstlisting}
\end{frame}

\begin{frame}[fragile]{Итерация по списку}
У списка нет методов для доступа к элементам по индексу.\\
Можно использовать range-based for:
\begin{lstlisting}
using std::string;
std::list<string> l = {"One", "Two", "Three"};
for (string & s : l)
    std::cout << s << std::endl;
\end{lstlisting}
\end{frame}

\begin{frame}[fragile]{Итерация по списку}
Для более сложных операций нужно использовать \emph{итераторы}.
\begin{lstlisting}
std::list<string>::iterator i = l.begin();
for ( ; i != l.end(); ++i)
    if (*i == "Two")
        break;
l.erase(i);
\end{lstlisting}

Итератор списка можно перемещать в обоих направлениях:
\begin{lstlisting}    
auto last = l.end();
--last; // последний элемент
\end{lstlisting}
\end{frame}

\begin{frame}[fragile]{Шаблон \texttt{forward\_list}}
Односвязный список. В любом месте контейнера 
вставка и удаление производятся за $O(1)$.
\begin{itemize}
    \item \texttt{insert\_after} и \texttt{emplace\_after} вместо \texttt{insert} и \texttt{emplace},
    \item  \texttt{before\_begin, cbefore\_begin},
    \item {\tt push\_front, emplace\_front, pop\_front},
    \item {\tt splice\_after},
    \item {\tt merge, remove, remove\_if, reverse,\\ sort, unique}.
\end{itemize}

\begin{lstlisting}
std::forward_list<std::string> fl = {"One", "Two"};
fl.emplace_front("Zero");
fl.push_front("Minus one");
std::cout << fl.front() << std::endl;
\end{lstlisting}
\end{frame}

\begin{frame}[fragile]{Шаблон {\tt basic\_string}}
Контейнер для хранения символьных последовательностей.
\begin{lstlisting}
typedef basic_string<char>	string;
typedef basic_string<wchar_t> 	wstring;
typedef basic_string<char16_t> 	u16string;
typedef basic_string<char32_t>	u32string;
\end{lstlisting}
\begin{itemize}
    \item Метод {c\_str()} для совместимости со старым кодом, 
    \item поддержка неявных преобразований с \langc-строками,
    \item \texttt{operator[], at},
    \item \texttt{reserve, capacity, shrink\_to\_fit},
    \item \texttt{append, operator+, operator+=},
    \item \texttt{substr, replace, compare,} 
    \item \texttt{find, rfind, find\_first\_of,\\ find\_first\_not\_of,
    find\_last\_of,\\ find\_last\_not\_of} (в терминах {\em индексов})
\end{itemize}
\end{frame}

\begin{frame}{Адаптеры и псевдоконтейнеры}
    Адаптеры:
    \begin{itemize}
        \item {\tt stack} —
            реализация интерфейса стека.

        \item {\tt queue} —
            реализация интерфейса очереди.

        \item {\tt priority\_queue} —
            очередь с приоритетом на куче.
    \end{itemize}
    
    Псевдо-контейнеры:
    \begin{itemize}
        \item {\tt vector<bool>}
        \begin{itemize}
            \item ненастоящий контейнер (не хранит {\tt bool}-ы),
            \item использует proxy-объекты.
        \end{itemize}
        \item {\tt bitset}\\
            Служит для хранения битовых масок.\\ Похож на
            {\tt vector<bool>}.
                        
        \item {\tt valarray}\\
            Шаблон служит для хранения числовых массивов
            и оптимизирован для достижения повышенной\\ вычислительной
            производительности. 
    \end{itemize}
\end{frame}

\begin{frame}[fragile]{Ещё о {\tt vector}}
\small\setlength{\baselineskip}{0.6em}
 \begin{itemize}
     \item Самый универсальный последовательный контейнер. 

     \item Во многих случаях самый эффективный.
         
     \item Предпочитайте \texttt{vector} другим контейнерам.
 
     \item Интерфейс построен на итераторах, а не на индексах.
 
     \item Итераторы ведут себя как указатели.
\end{itemize}

\medskip

\end{frame}

\begin{frame}[fragile]{Общие сведения}
Ассоциативные контейнеры делятся на две группы:
\begin{itemize}
\item \textit{упорядоченные} (требуют отношение порядка), 
\item \textit{неупорядоченные} (требуют хеш-функцию).
\end{itemize}
\begin{block}{Общие методы}
\begin{enumerate}
    \item {\tt find} по ключу,
    \item {\tt count} по ключу,
    \item {\tt erase} по ключу.
\end{enumerate}
\end{block}
\end{frame}

\begin{frame}[fragile]{Шаблоны {\tt set} и {\tt multiset}}
\texttt{set} хранит упорядоченное множество (дерево поиска).\\ 
Операции добавления, удаления и поиска работают за $O(\log n)$.
Значения в \texttt{set} — неизменяемые.
\begin{itemize}
    \item {\tt lower\_bound, upper\_bound, equal\_range}.
\end{itemize}

\begin{lstlisting}
std::set<int> primes = {2, 3, 5, 7, 11};
// дальнейшее заполнение
if (primes.find(173) != primes.end())
    std::cout << 173 << " is prime\n";
    
// std::pair<iterator, bool>
auto [it, inserted] = primes.insert(3);
\end{lstlisting}
\end{frame}

\begin{frame}[fragile]{Шаблоны {\tt set} и {\tt multiset}}
В \texttt{multiset} хранится упорядоченное мультимножество.
\vspace{-1mm}
\begin{lstlisting}
std::multiset<int> fib = {0, 1, 1, 2, 3, 5, 8};
// iterator
auto it = fib.insert(13);
// pair<iterator, iterator>
auto [it_begin, it_end] = fib.equal_range(1);
\end{lstlisting}
\end{frame}

\begin{frame}[fragile]{Шаблоны {\tt map} и {\tt multimap}}
Упорядоченное отображение (дерево поиска по ключу).\\ 
Операции добавления, удаления и поиска работают за $O(\log n)$.
\begin{lstlisting}
    using value_type = std::pair<const Key, T>;
\end{lstlisting}\vspace{-1mm}

\begin{itemize}
    \item {\tt lower\_bound,} {\tt upper\_bound,} {\tt equal\_range},
    \item {\tt operator[]}, {\tt at}.
\end{itemize}\vspace{-1mm}
\begin{lstlisting}
std::map<std::string, int> phonebook;
phonebook.emplace("Marge", 2128506);
phonebook.emplace("Lisa",  2128507);
phonebook.emplace("Bart",  2128507);
// std::map<string,int>::iterator
auto it = phonebook.find("Maggie");
if (it != phonebook.end())
    std::cout << "Maggie: " << it->second << "\n";
\end{lstlisting}
\end{frame}

\begin{frame}[fragile]{Особые методы {\tt map}: {\tt operator[]} и {\tt at}}
\begin{lstlisting}
if (auto it = phonebook.find("Marge"); 
        it != phonebook.end())
     it->second = 5550123;
else 
    phonebook.emplace("Marge", 5550123);
// или
phonebook["Marge"] = 5550123;
\end{lstlisting}
\end{frame}

\begin{frame}[fragile]{Особые методы {\tt map}: {\tt operator[]} и {\tt at}}
Метод {\tt \code{operator}[]}:
\begin{enumerate}
    \item работает только с неконстантным map,
    \item требует наличие у {\tt T} конструктора по умолчанию,
    \item работает за $O(\log n)$
    (не стоит использовать {\tt map} как массив).
\end{enumerate}
Метод {\tt at}:
\begin{enumerate}
    \item генерирует ошибку времени выполнения,\\
         если такой ключ отсутствует,
    \item работает за $O(\log n)$.
\end{enumerate}
\end{frame}

\begin{frame}[fragile]{Использование собственного компаратора}
\begin{lstlisting}
struct P { string name; string surname; };

bool operator<(P const& a, P const& b) {
    return a.name < b.name || 
          (a.name == b.name && a.surname < b.surname);
}
// уникальны по сочетанию имя + фамилия
std::set<P> s1;

struct PComp {
    bool operator()(P const& a, P const& b) const {
        return a.surname < b.surname;
    }
};
// уникальны по фамилии
std::set<P, PComp> s2;
\end{lstlisting}
\end{frame}


\begin{frame}[fragile]{Шаблоны {\tt unordered\_set} и {\tt unordered\_multiset}}
\texttt{unordered\_set} хранит множество как хеш-таблицу.\\ 
Операции добавления, удаления и поиска за $O(1)$ в среднем.
Значения в \texttt{unordered\_set} — неизменяемые.
\begin{itemize}
	\item {\tt equal\_range}, {\tt reserve},
	\item методы для работы с хеш-таблицей.
\end{itemize}\vspace{-1mm}

\begin{lstlisting}
unordered_set<int> primes = {2, 3, 5, 7, 11};
// дальнейшее заполнение
if (primes.find(173) != primes.end())
    std::cout << 173 << " is prime\n";
    
// std::pair<iterator, bool>
auto [it, inserted] = primes.insert(3);
\end{lstlisting}
В \texttt{unordered\_multiset} хранится мультимножество.
\end{frame}

\begin{frame}[fragile]{Шаблоны {\tt unordered\_map} и {\tt unordered\_multimap}}
Хранит отображение как хеш-таблицу.\\ 
Операции добавления, удаления и поиска за $O(1)$ в среднем.

\begin{itemize}
	\item {\tt equal\_range}, {\tt reserve}, {\tt operator[]}, {\tt at},
	\item методы для работы с хеш-таблицей.
\end{itemize}\vspace{-1mm}
\begin{lstlisting}
unordered_map<std::string, int> phonebook;
phonebook.emplace("Marge", 2128506);
phonebook.emplace("Lisa",  2128507);
phonebook.emplace("Bart",  2128507);

// unordered\_map<string,int>::iterator
auto it = phonebook.find("Maggie");
if (it != phonebook.end())
    std::cout << "Maggie: " << it->second << "\n";
\end{lstlisting}
\end{frame}

\begin{frame}[fragile]{Использование собственной хеш-функции}
\begin{lstlisting}
struct P { string name; string surname; };
bool operator==(P const& a, P const& b) { 		
    return a.name == b.name 
        && a.surname == b.surname; 
}
namespace std {
    template <> struct hash<P> {
        size_t operator()(P const& p) const {
              hash<string> h;
              return h(p.name) ^ h(p.surname);
        }
    };
}
// уникальны по сочетанию имя + фамилия
unordered_set<P> s;
\end{lstlisting}
\end{frame}

\begin{frame}[fragile]{Использование собственной хеш-функции}
    Общие рекомендации:
\begin{itemize}
        \item По возможности лучше не писать свои хеш-функции.
        \item Для комбинирования лучше использовать специальные функции.
        \item Пример реализации boost::hash\_combine:
\begin{lstlisting}
template <typename T>
inline void hash_combine(
    size_t seed,
    const T & v)
{
    std::hash<T> hasher;
    seed ^= hasher(v) + 0x9e3779b9
            + (seed<<6) + (seed>>2);
}
\end{lstlisting}
\end{itemize}
\end{frame}

\end{document}

 
