\documentclass{beamer}
\usepackage{csc}
\title{Лекция 3. Структуры данных и классы}

\date{
   \textbf{ИТМО JB}\\
   21 сентября 2017 \\
   Санкт-Петербург
}

\begin{document}
\begin{frame} 
  \titlepage
\end{frame}

\begin{frame}[fragile]{Зачем группировать данные?}
    Какая должна быть сигнатура у функции, которая вычисляет длину отрезка на
    плоскости?
    \begin{lstlisting}
double length(double x1, double y1, 
              double x2, double y2);
    \end{lstlisting}
    А сигнатура функции, проверяющей пересечение отрезков?
    \begin{lstlisting}
bool intersects(double x11, double y11, 
                double x12, double y12,
                double x21, double y21, 
                double x22, double y22,
                double * xi, double * yi);
    \end{lstlisting}
Координаты точек являются логически связанными данными, которые всегда передаются
вместе.\\
Аналогично связанны координаты точек отрезка.
\end{frame}

\begin{frame}[fragile]{Структуры}
    Структуры~--- это способ синтаксически (и физически)
    сгруппировать логически связанные данные.
    \begin{lstlisting}
struct Point {
    double x;
    double y;
};
struct Segment {
    Point p1;
    Point p2;
};

double length(Segment s);  

bool intersects(Segment s1, 
                Segment s2, Point * p);
    \end{lstlisting}
\end{frame}

\begin{frame}[fragile]{Работа со структурами}
    Доступ к полям структуры осуществляется через
    оператор '\code{.}':
    \begin{lstlisting}
#include <cmath>

double length(Segment s) {
    double dx = s.p1.x - s.p2.x;
    double dy = s.p1.y - s.p2.y;
    return sqrt(dx * dx + dy * dy);
}
    \end{lstlisting}
    Для указателей на структуры используется оператор '\code{->}'.
    \begin{lstlisting}
double length(Segment * s) {
    double dx = s->p1.x - s->p2.x;
    double dy = s->p1.y - s->p2.y;
    return sqrt(dx * dx + dy * dy);
}
    \end{lstlisting}
\end{frame}

\begin{frame}[fragile]{Инициализация структур}
    Поля структур можно инициализировать подобно массивам:
    \begin{lstlisting}
Point p1 = { 0.4, 1.4 };
Point p2 = { 1.2, 6.3 };
Segment s = { p1, p2 };
    \end{lstlisting}
    Структуры могут хранить переменные разных типов.
    \begin{lstlisting}
struct IntArray2D {
    size_t a;
    size_t b;
    std::vector<int> data;
};
    \end{lstlisting}
    \begin{lstlisting}
IntArray2D a = {n, m, create_array2d(n, m)};
    \end{lstlisting}
\end{frame}

\begin{frame}[fragile]{POD типы}
    \code{POD-типы} в языке \langcpp это аббревиатура от Plain Old Data, 
    что можно трактовать как простые данные в стиле \langc.\\
    К POD типам относятся:
    \begin{itemize}
        \item все встроенные арифметические типы (включая wchar\_t и bool);
        \item перечисления, т.е. типы, объявленные с помощью ключевого слова enum;
        \item указатели;
        \item POD-структуры (struct или class) и POD-объединения (union).
    \end{itemize}
\end{frame}

\begin{frame}[fragile]{POD типы}
    Чтобы структура была POD-типом, нужно выполнение следующих условий:
    \begin{itemize}
        \item не иметь пользовательских конструкторов, 
            деструктора или копирующего оператора присваивания;
        \item не иметь базовых классов;
        \item не иметь виртуальных функций;
        \item не иметь защищенных (protected) или закрытых (private) нестатических членов данных;
        \item не иметь нестатических членов данных не-POD-типов (или массивов из таких типов), а также ссылок.
    \end{itemize}
\end{frame}

\begin{frame}[fragile]{Structure binding}
    \begin{itemize}
        \item Определим простейшую POD структуру:
            \begin{lstlisting}
struct Point {
    int x;
    int y;
};
            \end{lstlisting}
        \item Structure binding:
            \begin{lstlisting}
Point p = { 1, 2 };
auto [x, y] = p; // C++17

foo(x); // equal to foo(p.x);
bar(y); // equal to bar(p.y);
            \end{lstlisting}
        \end{itemsize}
\end{frame}

\begin{frame}[fragile]{Structure binding}
    Такие конструкции удобно использовать вместе с алгоритмами из стандартной библиотеки, 
        которые часто возвращают \code{std::pair} в качестве результата своей работы.
    \begin{lstlisting}
std::unordered_map<std::string, std::string> map;

auto [it, inserted] = map.emplace("key", "value");

if (inserted) {
    // do something with it
}
else {
    // do something else with it
}
    \end{lstlisting}
\end{frame}

\begin{frame}[fragile]{Методы}
    Метод — это функция, определённая внутри структуры.
\small
    \begin{lstlisting}
struct Segment {
    Point p1;
    Point p2;
    double length() {
        double dx = p1.x - p2.x;
        double dy = p1.y - p2.y;
        return sqrt(dx * dx + dy * dy);
    }
};
int main() {
    Segment s = { { 0.4, 1.4 }, { 1.2, 6.3 } };
    cout << s.length() << endl;
    return 0;
}
    \end{lstlisting}
\end{frame}

\begin{frame}[fragile]{Методы}
    Методы реализованы как функции с неявным параметром \code{this},
    который указывает на текущий экземпляр структуры.
    \small
    \begin{lstlisting}
struct Point 
{
    double x;
    double y;

    void shift(/* Point * this, */
               double x, double y) {
        this->x += x;
        this->y += y;        
    }
};
    \end{lstlisting}
\end{frame}

\begin{frame}[fragile]{Методы: объявление и определение}
    Методы можно разделять на объявление и определение:
    \begin{lstlisting}
struct Point 
{
    double x;
    double y;

    void shift(double x, double y);
};
    \end{lstlisting}

    \begin{lstlisting}
void Point::shift(double x, double y) 
{
    this->x += x;
    this->y += y;
}
    \end{lstlisting}
\end{frame}

\begin{frame}[fragile]{Конструкторы}
    Конструкторы — это методы для инициализации структур.
    \begin{lstlisting}
struct Point {
    Point() { 
        x = y = 0;
    }
    Point(double x, double y) {
        this->x = x;
        this->y = y;
    }
    double x;
    double y;
};
    \end{lstlisting}
    \begin{lstlisting}
Point p1;
Point p2(3,7);
    \end{lstlisting}
\end{frame}

\begin{frame}[fragile]{Список инициализации}
    Список инициализации позволяет проинициализировать поля до входа в
    конструктор.
    \begin{lstlisting}
struct Point {
    Point() : x(0), y(0) 
    {}
    Point(double x, double y) : x(x), y(y) 
    {}

    double x;
    double y;
};
    \end{lstlisting}
Инициализации полей в списке инициализации
происходит в {\em порядке объявления полей} в структуре.
\end{frame}

\begin{frame}[fragile]{Значения по умолчанию}
    \begin{itemize}
        \item Функции могут иметь значения параметров {\em по умолчанию}.
        \item Значения параметров по умолчанию нужно указывать в {\em объявлении
            функции}.
            \begin{lstlisting}
struct Point {
    Point(double x = 0, double y = 0) 
        : x(x), y(y)
    {}
    double x;
    double y;
};
            \end{lstlisting}
            \begin{lstlisting}
Point p1;
Point p2(2);
Point p3(3,4);
            \end{lstlisting}
    \end{itemize}
\end{frame}

\begin{frame}[fragile]{Значения по умолчанию}
    \begin{itemize}
        \item Определим структуру:
            \begin{lstlisting}
struct Point {
    int x;
    int y;
};
            \end{lstlisting}
        \item Попробуем инициализировать ее значения:
            \begin{lstlisting}
Point p1 = { 1, 2 }; // ok
Point p2; // undefined
            \end{lstlisting}
        \end{itemsize}
\end{frame}

\begin{frame}[fragile]{Значения по умолчанию}
    \begin{itemize}
        \item Исправим нашу структуру:
            \begin{lstlisting}
struct Point {
    int x = 0;
    int y = 0;
};
            \end{lstlisting}
        \item Попробуем инициализировать ее значения:
            \begin{lstlisting}
Point p1 = { 1, 2 }; // ok
Point p2; // now ok
            \end{lstlisting}
        \end{itemsize}
\end{frame}

\begin{frame}[fragile]{Конструкторы от одного параметра}
    Конструкторы от одного параметра задают {\em неявное}
    пользовательское преобразование:
\begin{lstlisting}
struct Segment {
    Segment() {}
    Segment(double length) 
        : p2(length, 0)
    {}
    Point p1;
    Point p2;
};
\end{lstlisting}
\begin{lstlisting}
    Segment s1;
    Segment s2(10);
    Segment s3 = 20;
\end{lstlisting}
\end{frame}

\begin{frame}[fragile]{Конструкторы от одного параметра}
    Для того, чтобы запретить {\em неявное} пользовательское
преобразование, используется ключевое слово \code{explicit}.
\begin{lstlisting}
struct Segment {
    Segment() {}
    explicit Segment(double length) 
        : p2(length, 0)
    {}
    Point p1;
    Point p2;
};
\end{lstlisting}
\begin{lstlisting}
    Segment s1;
    Segment s2(10);
    Segment s3 = 20; // error
\end{lstlisting}
\end{frame}

\begin{frame}[fragile]{Конструкторы от одного параметра}
Неявное пользовательское преобразование, 
задаётся также конструкторами, которые могут принимать один параметр.
\begin{lstlisting}
struct Point {
    explicit Point(double x = 0, double y = 0) 
        : x(x), y(y)
    {}
    double x;
    double y;
};
\end{lstlisting}
\begin{lstlisting}
    Point p1;
    Point p2(2);
    Point p3(3,4);
    Point p4 = 5; // error
\end{lstlisting}
\end{frame}

\begin{frame}[fragile]{Конструктор по умолчанию}
    Если у структуры нет конструкторов, то конструктор 
    без параметров, {\em конструктор по умолчанию},
    генерируется компилятором.
\begin{lstlisting}
struct Segment {
    Segment(Point p1, Point p2) 
        : p1(p1), p2(p2)
    {}
    Point p1;
    Point p2;
};
\end{lstlisting}
\begin{lstlisting}
    Segment s1; // error
    Segment s2(Point(), Point(2,1));
\end{lstlisting}
\end{frame}

\begin{frame}[fragile]{Конструктор по умолчанию}
    Существует возможность добавить {\em конструктор по умолчанию}.
    В таком случае компилятор сгенерирует конструктор самостоятельно.
\begin{lstlisting}
struct Segment {
    Segment() = default; // C++11
    Segment(Point p1, Point p2) 
        : p1(p1), p2(p2)
    {}

    Point p1;
    Point p2;
};
\end{lstlisting}
\begin{lstlisting}
    // ok if Point type has default constructor
    Segment s1;
\end{lstlisting}
\end{frame}

\begin{frame}[fragile]{Удаление конструктора}
    Существует возможность удалить {\em конструктор по умолчанию}.
\begin{lstlisting}
struct Segment {
    Segment() = delete;

    Point p1;
    Point p2;
};
\end{lstlisting}
\begin{lstlisting}
    Segment s1; // error
    Segment s2(Point(), Point(2,1)); // error
    Segment s2{ Point(), Point(2,1) }; // ok until C++20
\end{lstlisting}
\end{frame}

\begin{frame}[fragile]{Особенности синтаксиса \langcpp}

    {\em ``Если что-то похоже на объявление функции, то это
    и есть объявление функции.''}
\begin{lstlisting}
struct Point {
    explicit Point(double x = 0, double y = 0) 
        : x(x), y(y) {}
    double x;
    double y;
};
\end{lstlisting}
\begin{lstlisting}
    Point p1;   // определение переменной
    Point p2(); // объявление функции

    double k = 5.1;
    Point p3(int(k)); // объявление функции
    Point p4((int)k); // определение переменной
\end{lstlisting}
\end{frame}

\begin{frame}[fragile]{Деструктор}
    Деструктор — это метод, который вызывается при удалении структуры,
    генерируется компилятором.
\begin{lstlisting}
struct Point {
    Point() {
        std::cout << "A" << std::endl;
    }

    ~Point() {
        std::cout << "B" << std::endl;
    }
};

void foo() { Point p; }

foo(); // prints AB
\end{lstlisting}
\end{frame}

\begin{frame}[fragile]{Время жизни}{}
    {\em Время жизни}~--- это временной интервал 
    между вызовами конструктора и деструктора.
\begin{lstlisting}
void foo() {
    Point p1; // создание p1
    Point p2; // создание p2

    {
        Point p3; // создание p3
    } // удаление p3
} // удаление p2, потом p1
\end{lstlisting}
Деструкторы переменных на стеке вызываются в обратном
порядке (по отношению к порядку вызова конструкторов).
\end{frame}

\begin{frame}[fragile]{Объекты и классы}
    \small
    \begin{itemize}
        \item Структуру с методами, конструкторами и деструктором
            называют {\em классом}.
        \item Экземпляр (значение) класса называется {\em объектом}.
    \end{itemize}
\end{frame}

\begin{frame}[fragile]{Модификаторы доступа}{}
    Модификаторы доступа позволяют ограничивать 
    доступ к методам и полям класса.
\begin{lstlisting}
struct Point {
    explicit Point(double x = 0, double y = 0) 
        : x(x), y(y) {}

    int x() { return x; }
    void set_x(int x_new) { x = x_new; }

    // y methods here
private:
    double x;
    double y;
};
\end{lstlisting}
\end{frame}

\begin{frame}[fragile]{Ключевое слово \code{class}}{}
    Ключевое слово \code{struct} можно заменить на 
    \code{class}, тогда поля и методы
    по умолчанию будут private.

\begin{lstlisting}
class Point {
    double x;
    double y;

public:
    explicit Point(double x = 0, double y = 0) 
        : x(x), y(y) {}

    int x() { return x; }
    void set_x(int x_new) { x = x_new; }

    // y methods here
};
\end{lstlisting}
\end{frame}

\begin{frame}[fragile]{Ключевое слово \code{class}}{}
\begin{lstlisting}
class Point {
public:
    explicit Point(double x = 0, double y = 0) 
        : x_(x), y_(y) {}

    int x() { return x_; }
    void set_x(int x_new) { x_ = x_new; }

    // y methods here

private:
    double x_;
    double y_;
};
\end{lstlisting}
\end{frame}

\begin{frame}[fragile]{Инварианты класса}{}
    \begin{itemize}
        \item Выделение {\em публичного интерфейса} позволяет
            поддерживать {\em инварианты класса}\\
            (сохранять
            данные объекта в согласованном состоянии).
\begin{lstlisting}
struct IntArray {
    ...
    size_t size_;
    int * data_; // массив размера size\_
};
\end{lstlisting}
        
        \item Для сохранения инвариантов класса:
            \begin{enumerate}
                \item все поля должны быть закрытыми, 
                \item публичные методы должны сохранять инварианты класса.
            \end{enumerate}

        \item Закрытие полей класса позволяет абстрагироваться 
            от способа хранения данных объекта.
    \end{itemize}
\end{frame}

\begin{frame}[fragile]{Публичный интерфейс}{}
\begin{lstlisting}
struct IntArray {
    ...
    void resize(size_t nsize) {
        int * ndata = new int[nsize];
        size_t n = nsize > size_ ? size_ : nsize;
        for (size_t i = 0; i != n; ++i)
            ndata[i] = data_[i];
        delete[] data_;
        data_ = ndata;
        size_ = nsize;
    }
private:
    size_t size_;
    int * data_;
};
\end{lstlisting}
\end{frame}

\begin{frame}[fragile]{Абстракция}{}
\begin{lstlisting}
struct IntArray {
public:
    explicit IntArray(size_t size) 
        : size_(size), data_(new int[size]) 
    {}
    ~IntArray() { delete[] data_;  }

    int & get(size_t i) { return data_[i]; }
    size_t size() { return size_; }

private:
    size_t size_;
    int * data_;
};
\end{lstlisting}
\end{frame}

\begin{frame}[fragile]{Абстракция}{}
\begin{lstlisting}
struct IntArray {
public:
    explicit IntArray(size_t size) 
        : data_(new int[size + 1]) 
    {
        data_[0] = size;
    }
    ~IntArray() { delete[] data_;  }

    int & get(size_t i) { return data_[i + 1]; }
    size_t size() { return data_[0]; }

private:
    int * data_;
};
\end{lstlisting}
\end{frame}

\begin{frame}[fragile]{Константные методы}{}
    \begin{itemize}
        \item Методы могут быть объявлены как \code{const}.
\begin{lstlisting}
struct S
{
    const int & ref() const {
        return data_;
    }
private:
    int data_;
};
\end{lstlisting}
        \item Такие методы не могут менять поля объекта.
        \item У константных объектов (через указатель или ссылку на константу)
            можно вызывать только константные методы.
        \item Внутри константных методов можно вызывать только константные методы.
    \end{itemize}
\end{frame}

\begin{frame}[fragile]{Две версии одного метода}{}
    \begin{itemize}
        \item Слово \code{const} является частью сигнатуры метода.
        \begin{lstlisting}
const int & S::ref() const { return data_; }
        \end{lstlisting}
        \item Можно определить две версии одного метода:
        \begin{lstlisting}
struct S
{
    const int & ref() const {
        return data_;
    }
    int & ref() {
        return data_;
    }
private:
    int data_;
};
        \end{lstlisting}
    \end{itemize}
\end{frame}

\begin{frame}[fragile]{Cинтаксическая и логическая константность}{}
    \begin{itemize}
    \item Синтаксическая константность~--- константные методы
        не могут менять поля (обеспечивается компилятором).
    \item Логическая константность~--- нельзя менять те данные, 
        которые определяют состояние объекта.
        \begin{lstlisting}
struct S 
{
    void foo() const {
        data_ = 10; // error
    }
private:
    int data_;
};
        \end{lstlisting}
    \end{itemize}
\end{frame}

\begin{frame}[fragile]{Ключевое слово \code{mutable}}{}
    Ключевое слово \code{mutable} позволяет определять поля,
    которые можно изменять внутри константных методов:
    \begin{lstlisting}
struct S 
{
    void foo() const {
        data_ = 10; // now ok
    }
private:
    mutable int data_;
};
    \end{lstlisting}
\end{frame}

\begin{frame}[fragile]{Копирование и перемещение объектов}{}
    Язык \langcpp позволяет определить поведение при копировании и перемещении объектов.
    \begin{lstlisting}
struct S 
{
    S() = default;
    // Копирующий конструктор.
    S(const S &) = default;
    // Перемещающий конструктор.
    S(S &&) = default;

    // Оператор копирующего присваивания.
    S & operator=(const S &) = default;
    // Оператор перемещающего присваивания.
    S & operator=(S &&) = default;
private:
    int data_;
};
    \end{lstlisting}
\end{frame}

\begin{frame}[fragile]{Копирование объектов}{}
    Определим конструктор и оператор копирования:
    \begin{lstlisting}
struct S 
{
    S() = default;
    S(const S & other)
        : data_(other.data)
    {}
    S & operator=(const S & other) {
        data_ = other.data;
        return *this;
    }
private:
    std::vector<int> data_ = decltype(data_)(100500);
};
    \end{lstlisting}
\end{frame}

\begin{frame}[fragile]{Копирование объектов}{}
    \begin{lstlisting}    
int main() {
    S s1;
    S s2(s1); // copy
    S s3 = s1; // copy
    s2 = s1; // operator=
}
    \end{lstlisting}
\end{frame}
    
\begin{frame}[fragile]{Копирование объектов}{}
    \begin{itemize}
        \item Если не определить конструктор копирования,
            то он сгенерируется компилятором.
        \item Если не определить оператор присваивания,
            то он тоже сгенерируется компилятором.
    \end{itemize}
\end{frame}

\begin{frame}[fragile]{Перемещение объектов}{}
    Определим конструктор и оператор перемещения:
    \begin{lstlisting}
struct S 
{
    S() = default;
    S(S && other)
        : data_(std::move(other.data_))
    {}
    S & operator=(S && other) {
        data_ = std::move(other.data_);
        return *this;
    }
private:
    std::vector<int> data_ = decltype(data_)(100500);
};
    \end{lstlisting}
\end{frame}

\begin{frame}[fragile]{Перемещение объектов}{}
    \begin{lstlisting}    
int main() {
    S s1;
    S s2(std::move(s1)); // move
    S s3 = std::move(s2); // move

    S s4;
    s4 = std::move(s3); // move operator=
    
    foo(s1); // undefined use-after-move
}
    \end{lstlisting}
\end{frame}
    
\begin{frame}[fragile]{Перемещение объектов}{}
    \begin{itemize}
        \item Если не определить конструктор перемещения,
            то он сгенерируется компилятором.
        \item Если не определить оператор перемещения,
            то он тоже сгенерируется компилятором.
    \end{itemize}
\end{frame}

\begin{frame}[fragile]{Правило 3/5/0}{}
    \begin{itemize}
        \item Правило 3. Если определен пользовательский деструктор, копирующий конструктор или копирующий \code{operator=},
            то следует определять все 3 сразу.
        \item Правило 5. Если определен пользовательский деструктор, копирующий конструктор или копирующий \code{operator=},  
            то их определение не позволяет компилятору сгенерировать методы для перемещения.
            Следует определять все 5 методов сразу.
        \item Правило 0. Если можно обойтись без пользовательских методов копирования/перемещения/деструкторов, то не следует их определять.
            Следуйте SRP - single responsibility principle.
    \end{itemize}
    Замечание: подробнее можно прочитать на cppreference или в cpp core guidelines.
\end{frame}

\end{document}
