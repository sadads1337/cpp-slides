\documentclass{beamer}
\usepackage{csc}
\title{Лекция 7. Шаблоны.}

\date{
   \textbf{ИТМО JB}\\
   19 октября 2021 \\
   Санкт-Петербург
}

\begin{document}
\begin{frame} 
  \titlepage
\end{frame}

\begin{frame}[fragile]{Глобальные переменные}
Объявление глобальной переменной:
    \begin{lstlisting}
extern int global;

void f () {
    ++global; 
}
    \end{lstlisting}

Определение глобальной переменной:
    \begin{lstlisting}
int global = 10;
    \end{lstlisting}

Проблемы глобальных переменных:
\begin{itemize}
    \item Масштабируемость.
    \item Побочные эффекты.
    \item Порядок инициализации.
\end{itemize}
\end{frame}

\begin{frame}[fragile]{Статические глобальные переменные}
    Статическая глобальная переменная — это глобальная переменная,
    доступная только в пределах модуля.\\[1em]
Определение: 
    \begin{lstlisting}
static int global = 10;

void f () {
    ++global; 
}
    \end{lstlisting}
                
Проблемы статических глобальных переменных:
\begin{itemize}
    \item Масштабируемость.
    \item Побочные эффекты.
\end{itemize}
\end{frame}

\begin{frame}[fragile]{Статические локальные переменные}
    Статическая локальная переменная — это глобальная переменная,
    доступная только в пределах функции.\\[1em]
Время жизни такой переменной~--- от первого вызова функции {\tt next}
до конца программы.
    \begin{lstlisting}
int next(int start = 0) {
    static int k = start;
    return k++;
}
    \end{lstlisting}
                
Проблемы статических локальных переменных:
\begin{itemize}
    \item Масштабируемость.
    \item Побочные эффекты.
\end{itemize}
\end{frame}

\begin{frame}[fragile]{Статические функции}
    Статическая функция,
    доступная только в пределах модуля.\\[1em]
Файл {\tt 1.cpp}:
    \begin{lstlisting}
static void test() {
    cout << "A\n";
}
    \end{lstlisting}
Файл {\tt 2.cpp}:
    \begin{lstlisting}
static void test() {
    cout << "B\n";
}
    \end{lstlisting}
Статические глобальные переменные и статические функции
проходят {\em внутреннюю линковку}.
\end{frame}

\begin{frame}[fragile]{Про линковку}
    \begin{itemize}
        \item Нас интересуют 2 типа линковки: \code{internal} и \code{external}.
        \item Внутренняя линковка проходит в рамках одной единицы трансляции (.cpp файла). \\
        \item Статические глобальные переменные и статические функции, функции в безымянном \code{namespace}
            проходят внутреннюю линковку.
    \end{itemize}
\end{frame}

\begin{frame}[fragile]{Статические поля класса}
    Статические поля класса — это глобальные переменные,
    определённые внутри класса.\\[1em]
Объявление:
\begin{lstlisting}
struct User {
    ...
private:
    static size_t instances_;
}; 
\end{lstlisting}
Определение:
\begin{lstlisting}
size_t User::instances_ = 0;
\end{lstlisting}
Для доступа к статическим полям не нужен объект.
\end{frame}

\begin{frame}[fragile]{Статические методы}
    Статические методы — это функции, определённые внутри класса и имеющие
    доступ к закрытым полям и методам.\\[1em]
Объявление:
\begin{lstlisting}
struct User {
    ...
    static size_t count() { return instances_; }
private:
    static size_t instances_;
}; 
\end{lstlisting}
Для вызова статических методов не нужен объект.
\begin{lstlisting}
    cout << User::count();
\end{lstlisting}
\end{frame}
    
\begin{frame}[fragile]{Ключевое слово \code{inline}}
    Советует компилятору встроить данную функцию.
\begin{lstlisting}
inline double square(double x) { return x * x; }
\end{lstlisting}
\begin{itemize}
    \item В месте вызова \code{inline}-функции должно 
        быть известно её определение.

    \item \code{inline} функции можно определять 
        в заголовочных файлах.

    \item Все функции, определённые внутри класса, являются
        \code{inline}.

    \item При линковке из всех версий \code{inline}-функции
        (т.е. её код из разных единиц трансляции)
        выбирается только одна.

    \item Все определения одной и той же \code{inline}-функции
        должны быть идентичными.

    \item \code{inline} — это совет компилятору, а не указ.
\end{itemize}
\end{frame}

\begin{frame}[fragile]{Правило одного определения}
Правило одного определения (One Definition Rule, ODR)
\begin{itemize}
    \item В пределах любой единицы трансляции сущности не могут иметь более одного определения. 
    \item В пределах программы глобальные переменные и не-\code{inline} функции не могут иметь больше
        одного определения.
    \item Классы и \code{inline} функции могут определяться в более чем одной единице
        трансляции, но определения обязаны совпадать.
\end{itemize}
\end{frame}

\begin{frame}[fragile]{Класс Singleton}
\begin{lstlisting}
struct Singleton {
    static Singleton & instance() {
           static Singleton s;
           return s;
    }

    Data & data() { return data_; }

private:
    Singleton() = default;
    Singleton(Singleton const&) = delete;
    Singleton& operator=(Singleton const&) = delete;

    Data data_;
};
\end{lstlisting}
\end{frame}

\begin{frame}[fragile]{Использование Singleton-а}{}
\begin{lstlisting}
    int main()
    {
        // первое обращение
        Singleton & s = Singleton::instance();
        Data d = s.data();

        // аналогично d = s.data();
        d = Singleton::instance().data();
        return 0;    
    }
\end{lstlisting}
\end{frame}

\begin{frame}[fragile]{Метапрограммирование}
\begin{itemize}
    \item {\bf Метапрограммированием} называют создание программ, \\которые порождают другие программы.
    
    \item Шаблоны \langcpp можно рассматривать как функциональный\\ язык
    для метапрограммирования.
    
    \item Метапрограммы \langcpp позволяют оперировать типами, \\шаблонами и compile-time значениями.
\end{itemize}
\end{frame}

\begin{frame}[fragile]{Метапрограммирование}
\begin{itemize}    
    \item Метапрограммирование в \langcpp можно применять\\ для широкого круга задач:
    \begin{itemize}
        \item compile-time вычисления,
        \item compile-time проверка ошибок,
        \item условная компиляция,
        \item генеративное программирование,
        \item \ldots
    \end{itemize}
    \item Для метапрограммирования существуют целые\\ библиотеки,
    например, MPL и Hana из boost.
\end{itemize}
    Замечание: сложные шаблоны существенно замедляют компиляцию.
\end{frame}

\begin{frame}[fragile]{Метафункции}
Метафункция~--- это шаблонный класс, 
который определяет \\имя типа \texttt{type} или целочисленную константу \texttt{value}.


\begin{itemize}
\item Аргументы метафункции~--- это аргументы шаблона.
\item Возвращаемое значение~--- это \texttt{type} или \texttt{value}.
\end{itemize}
\end{frame}

\begin{frame}[fragile]{Метафункции}
Метафункции могут возвращать типы:
\begin{lstlisting}
template<typename T> 
struct add_pointer
{
    using type = T *; 
};
\end{lstlisting}
и значения целочисленных типов:
\begin{lstlisting}
template<int N> 
struct square
{
    static int const value = N * N; 
};
\end{lstlisting}
\end{frame}

\begin{frame}[fragile]{Метафункции}
Для типов удобно использовать шаблонные алиасы имен:
\begin{lstlisting}
template<typename T>
using add_pointer_t = add_pointer<T>::type;
\end{lstlisting}
для значений аналогичные константы:
\begin{lstlisting}
template<int N> 
constexpr auto square_v = square<N>::value; 
\end{lstlisting}
\end{frame}

\begin{frame}[fragile]{Метафункции}
\begin{lstlisting}
template<typename T>
int foo(T * value) {
    constexpr auto is_pointer = std::is_same_v<
        int *,
        add_pointer_t<T>>;
    if constexpr (is_pointer) { return *value; }
    else { return square_v<10>; }
}

int main() {
    int a = 1;
    float b = 1.f;
    foo(&a); // 1
    foo(&b); // 100
}
\end{lstlisting}

\end{frame}


\begin{frame}[fragile]{Вычисления в compile-time}
\begin{lstlisting}
template<int N> 
struct Fact {
    static int const value
        = N * Fact<N - 1>::value;
};

template<> 
struct Fact<0> {
    static int const value = 1;
};

int main() 
{
    std::cout << Fact<10>::value << std::endl;
}
\end{lstlisting}
\end{frame}

\begin{frame}[fragile]{Вычисления в compile-time}
    Это вычисление можно реализовать через \code{constexpr} функцию.
\begin{lstlisting}
constexpr int fact(int N) {
    if (N == 0) { return 1; }
    return N * fact(N - 1);
}

int main() 
{
    constexpr auto a = fact(10);
    std::cout << a << std::endl;
}
\end{lstlisting}
\end{frame}

\begin{frame}[fragile]{Вычисления в compile-time}
    Подход без рекурсии.
\begin{lstlisting}
constexpr int fact(int N) {
    int result = 1;
    for (auto i = 1; i <= N; ++i)
    {
        result *= i;
    }
    return result;
}

int main() 
{
    constexpr auto a = fact(10);
    std::cout << a << std::endl;
}
\end{lstlisting}
\end{frame}

%template<int N> struct Fib {
%    constexpr static int 
%        value = Fib<N - 1>::value + Fib<N - 2>::value;
%};
%template<> struct Fib<0> {
%    static const int value = 0;
%};
%template<> struct Fib<1> {
%    static const int value = 1;
%};
%int main() {
%    Fib <10>::value << std::endl;
%}

\begin{frame}[fragile]{Определение списка}
Шаблоны позволяют определять алгебраические типы данных.
\begin{lstlisting}
// определяем список
template <typename ... Types>
struct TypeList; 

// специализация по умолчанию
template <typename H, typename... T>
struct TypeList<H, T...> 
{
    using Head = H;
    using Tail = TypeList<T...>;
};

// специализация для пустого списка
template <>
struct TypeList<> { };
\end{lstlisting}
\end{frame}

\begin{frame}[fragile]{Длина списка}
\begin{lstlisting}
// вычисление длины списка
template<typename TL>
struct Length {
    static int const value = 1 +
        Length<typename TL::Tail>::value;
};

template<>
struct Length<TypeList<>> {
    static int const value = 0;
};

int main() {
    using TL = TypeList<double, float, int, char>;
    std::cout << Length<TL>::value << std::endl;
}
\end{lstlisting}
\end{frame}

\begin{frame}[fragile]{Операции со списком}
    Добавление элемента в начало списка:
\begin{lstlisting}
template<typename H, typename TL>
struct Cons;

template<typename H, typename... Types>
struct Cons<H, TypeList<Types...>> {
    using type = TypeList<H, Types...>;
};
\end{lstlisting}
\end{frame}

\begin{frame}[fragile]{Операции со списком}
    Конкатенация списков:
\begin{lstlisting}
template<typename TL1, typename TL2>
struct Concat;

template<typename... Ts1, typename... Ts2>
struct Concat<TypeList<Ts1...>, TypeList<Ts2...>>
{
    using type = TypeList<Ts1..., Ts2...>;
};
\end{lstlisting}
\end{frame}

\begin{frame}[fragile]{Вывод списка}
\begin{lstlisting}
// вывод списка в поток os
template<typename TL>
void printTypeList(std::ostream & os) {
    os << typeid(typename TL::Head).name() << '\n';
    printTypeList<typename TL::Tail>(os);
};

// вывод пустого списка
template<>
void printTypeList<TypeList<>>(std::ostream &) {}

int main() {
    using TL = TypeList<double, float, int, char>;
    printTypeList<TL>(std::cout);
}
\end{lstlisting}
\end{frame}

\begin{frame}[fragile]{Генерация классов}
\begin{lstlisting}
struct A { 
    void foo() {std::cout << "struct A\n";} 
};
struct B { 
    void foo() {std::cout << "struct B\n";} 
};
struct C { 
    void foo() {std::cout << "struct C\n";} 
};

using Bases = TypeList<A, B, C>;
\end{lstlisting}
\end{frame}

\begin{frame}[fragile]{Генерация классов}
\begin{lstlisting}
template<typename TL>
struct inherit;

template<typename... Types>
struct inherit<TypeList<Types...>> : Types... {};

struct D : inherit<Bases> { };
\end{lstlisting}
\end{frame}


\begin{frame}[fragile]{Генерация классов}
\begin{lstlisting}
struct D : inherit<Bases> {
    void foo() { foo_impl<Bases>(); } 
    template<typename L> void foo_impl();
};
template<typename L> 
inline void D::foo_impl() {
    // приводим this к указателю на базу из списка
    static_cast<typename L::Head *>(this)->foo();
    
    // рекурсивный вызов для хвоста списка
    foo_impl<typename L::Tail>();
}
template<> 
inline void D::foo_impl<TypeList<>>() {}
\end{lstlisting}
\end{frame}

%\begin{frame}[fragile]{CRTP: Polymorphic copy construction}
%\begin{lstlisting}
%// Base class has a pure virtual function for cloning
%struct Shape {
%    virtual ~Shape() {}
%    virtual Shape *clone() const = 0;
%};

%// This CRTP class implements clone() for Derived
%template <typename Derived>
%struct Shape_CRTP : public Shape {
%    virtual Shape *clone() const {
%        return new Derived(static_cast<Derived const&>(*this));
%    }
%};
% 
%// Nice macro which ensures correct CRTP usage
%#define Derive_Shape_CRTP(Type) struct Type: Shape_CRTP<Type>
% 
%// Every derived class inherits from Shape\_CRTP instead of Shape
%Derive_Shape_CRTP(Square) {};
%Derive_Shape_CRTP(Circle) {};
%\end{lstlisting}
%\end{frame}

\begin{frame}[fragile]{SFINAE}
 SFINAE = Substitution Failure Is Not An Error.\\
 Ошибка при подстановке шаблонных параметров не является сама по себе ошибкой.
\begin{lstlisting}
// ожидает, что у типа T определён 
// вложенный тип value\_type
template<class T>
void foo(typename T::value_type * v);

// работает с любым типом
template<class T>
void foo(T t);
\end{lstlisting}
\begin{lstlisting}
// при инстанциировании первой перегрузки
// происходит ошибка (у int нет value\_type), 
// но это не приводит к ошибке компиляции
foo<int>(0);
\end{lstlisting}
\end{frame}

\begin{frame}[fragile]{Полная специализация шаблонов: классы}
\small
    \begin{lstlisting}
template<class T>
struct Array { ... };

template<>
struct Array<bool> {
    static unsigned const BITS = 8 * sizeof(unsigned);
    explicit Array(size_t size)
        : size_(size)
        , data_(new unsigned[size_ / BITS + 1])
    {}
    bool operator[](size_t i) const {
        return data_[i / BITS] & (1 << (i % BITS)); 
    }
private:
    size_t  size_;
    unsigned *  data_;
};
    \end{lstlisting}
\end{frame}

\begin{frame}[fragile]{Полная специализация шаблонов: функции}
\small
    \begin{lstlisting}
template<class T>
void swap(T & a, T & b) {
    T tmp(a);
    a = b;
    b = tmp;
}

template<>
void swap<Database>(Database & a, Database & b) {
    a.swap(b);
}

template<class T>
void swap(Array<T> & a, Array<T> & b) {
    a.swap(b);
}
    \end{lstlisting}
\end{frame}

\begin{frame}[fragile]{Специализация шаблонов и перегрузка}
\small
    \begin{lstlisting}
template<class T>
void foo(T, T) { std::cout << "same"; }

template<class T, class V>
void foo(T, V) { std::cout << "different"; }

template<>
void foo<int, int>(int, int) { std::cout << "both int"; } 

int main() {
    foo(3, 4); // same (not both int)
    foo(3., 4); // different
    return 0;    
}
    \end{lstlisting}
\end{frame}

\begin{frame}[fragile]{Частичная специализация шаблонов}
\small
    \begin{lstlisting}
template<class T>
struct Array {
    T & operator[](size_t i) { return data_[i]; }
    ...
};

template<class T>
struct Array<T *> {
    explicit Array(size_t size)
        : size_(size)
        , data_(new T *[size_])
    {}
    T & operator[](size_t i) { return *data_[i]; }
private:
    size_t  size_;
    T **    data_;
};
    \end{lstlisting}
\end{frame}

\begin{frame}[fragile]{Как проверить наличие родственных связей?}
\small
\begin{lstlisting}
using YES = char;
struct NO { YES m[2]; };

template<class B, class D>
struct is_base_of {
    static YES test(B * );
    static NO  test(...);
    
    static bool const value = 
        sizeof(YES) == sizeof(test((D *)0));
};

template<class D> 
struct is_base_of<D, D> {
    static bool const value = false;
};
\end{lstlisting}
\end{frame}

\begin{frame}[fragile]{Как проверить наличие родственных связей?}
\small
\begin{lstlisting}
namespace details {
    template <typename B>
    std::true_type test(B*);
    template <typename>
    std::false_type test(void*);
 
    template <typename, typename>
    auto is_base_of(...) -> std::true_type;
    template <typename B, typename D>
    auto is_base_of(int) -> decltype(
        test<B>(static_cast<D*>(nullptr)));
}
\end{lstlisting}
\end{frame}

\begin{frame}[fragile]{Как проверить наличие родственных связей?}
\small
\begin{lstlisting}
template <typename B, typename D>
struct is_base_of
    : std::integral_constant<
        bool,
        std::is_class_v<B> && std::is_class_v<D> &&
        decltype(details::is_base_of<B, D>(0))::value
    >{};

class A {};
class B : A {};
 
int main() {
    std::cout << std::is_base_of<A, B>::value; // 1
    std::cout << std::is_base_of<B, A>::value; // 2
}
\end{lstlisting}
\end{frame}


\begin{frame}[fragile]{Как определить наличие метода?}
\begin{lstlisting}
struct A {  void foo() { std::cout << "struct A\n"; } };
struct B {  }; // нет метода foo()
struct C {  void foo() { std::cout << "struct C\n"; } };
\end{lstlisting}

\begin{lstlisting}
template<typename L> 
inline void D::foo_impl() 
{
    // приводим this к указателю на базу из списка
    static_cast<typename L::Head *>(this)->foo();
    
    // рекурсивный вызов для хвоста списка
    foo_impl<typename L::Tail>();
}
\end{lstlisting}
\end{frame}

\begin{frame}[fragile]{Используем SFINAE}
\begin{lstlisting}
template<class T>
struct is_foo_defined
{
    // обёртка, которая позволит проверить
    // наличие метода foo с заданой сигнатурой
    template<class Z, void (Z::*)() = &Z::foo>
    struct wrapper {};

    template<class C>
    static std::true_type check(wrapper<C> * p);
    template<class C>
    static std::false_type check(...);

    static bool const value = std::is_same_v<
        std::true_type,
        decltype(check<T>(0))>;
};
\end{lstlisting}
\end{frame}

\begin{frame}[fragile]{Проверяем наличие метода}

\begin{lstlisting}
template<class L> 
void foo_impl() 
{
    using Head = typename L::Head;
    constexpr bool has_foo = 
        is_foo_defined<Head>::value;
    if constexpr (has_foo) {
        // call foo
    }
    foo_impl<typename L::Tail>();
}
\end{lstlisting}
\end{frame}

\begin{frame}[fragile]{Проверяем наличие метода}

C++20 все сильно упрощает.

\begin{lstlisting}
template<class T>
std::string optionalToString(T* obj)
{
    constexpr bool has_str = requires(const T& t) {
        t.toString();
    };

    if constexpr (has_str)
        return obj->toString();
    else
        return "toString not defined";
}
\end{lstlisting}
\end{frame}

\begin{frame}[fragile]{{\tt std::enable\_if}}
\begin{lstlisting}
namespace std {
    template<bool B, class T = void> 
    struct enable_if {};

    template<class T>
    struct enable_if<true, T> { using type = T; };
}
\end{lstlisting}
\begin{lstlisting}
template<class T>
typename std::enable_if_t<std::is_integral_v<T>, T>
    div2(T t) { return t >> 1; }

template<class T>
typename std::enable_if_t<std::is_floating_point_v<T>, T>
    div2(T t) { return t / 2.0; }
\end{lstlisting}
\end{frame}

\begin{frame}[fragile]{{\tt std::enable\_if}}
\begin{lstlisting}
template<class T>
T div2(T t, typename std::enable_if_t<
    std::is_integral_v<T>, T> * = 0)
{ return t >> 1; }

template<class T, class E = typename std::enable_if_t<
    std::is_floating_point_v<T>::value, T>>
T div2(T t) 
{ return t / 2.0; }
\end{lstlisting}

\begin{lstlisting}
template<class T, class E = void> 
class A; 
 
template<class T>
class A<T, typename std::enable_if_t<
    std::is_integral_v<T>>> 
{};
\end{lstlisting}
\end{frame}

\begin{frame}[fragile]{Нетиповые шаблонные параметры}
\small
Параметрами шаблона могут быть целочисленные значения.
    \begin{lstlisting}
template<class T, size_t N, size_t M>
struct Matrix {
    ...
    T & operator()(size_t i, size_t j) 
    { return data_[M * j + i]; }
private:
    T data_[N * M];
};

template<class T, size_t N, size_t M, size_t K>
Matrix<T, N, K> operator*(Matrix<T, N, M> const& a, 
                          Matrix<T, M, K> const& b);
    \end{lstlisting}
\end{frame}

\begin{frame}[fragile]{Нетиповые шаблонные параметры}
\small
Параметрами шаблона могут быть указатели/ссылки на значения с внешней линковкой.
    \begin{lstlisting}
// log - это глобальная переменная
template<ofstream & log>
struct FileLogger { ... };
    \end{lstlisting}
\end{frame}

\begin{frame}[fragile]{Шаблонные параметры~--- шаблоны}
Параметрами шаблона могут быть шаблоны.
\small
    \begin{lstlisting}
// int --> string
string toString( int i );

// работает только с Array<>
Array<string> toStrings( Array<int> const& ar ) {
    Array<string> result(ar.size());
    for (size_t i = 0; i != ar.size(); ++i)
        result.get(i) = toString(ar.get(i));
    return result;
}
    \end{lstlisting}
\end{frame}

\begin{frame}[fragile]{Шаблонные параметры~--- шаблоны}
\small
    \begin{lstlisting}
// от контейнера требуются:
// - конструктор от size
// - методы size() и get()
template<template <class> class Container>
auto toStrings(Container<int> const& c) {
    Container<string> result(c.size());
    for (size_t i = 0; i != c.size(); ++i)
        result.get(i) = toString(c.get(i));
    return result;
}
    \end{lstlisting}
\end{frame}

\begin{frame}[fragile]{Компиляция шаблонов}
\small
    \begin{itemize}
        \item Шаблон независимо компилируется для каждого значения шаблонных
            параметров.

        \item Компиляция ({\em инстанциирование}) шаблона происходит в точке первого использования~---
            {\em точке инстанциирования шаблона}.

        \item Компиляция шаблонов классов~— ленивая, компилируются только те методы,
            которые используются.
        
        \item В точке инстанциирования шаблон должен быть полностью определён.

        \item Шаблоны следует определять в заголовочных файлах.

        \item Все шаблонные функции (свободные функции и методы) являются
            \code{inline}.

        \item В разных единицах трансляции инстанциирование происходит
            независимо.
    \end{itemize}
\end{frame}

\begin{frame}[fragile]{Резюме про шаблоны}
\small
    \begin{itemize}
        \item
            Большие шаблонные классы следует разделять на два заголовочных файла:
            объявление (\texttt{array.hpp}) и определение (\texttt{array\_impl.hpp}).

        \item Частичная специализация и шаблонные параметры по умолчанию
            есть только у шаблонов классов.

        \item Вывод шаблонных параметров есть только у шаблонов функций.

        \item Предпочтительно использовать перегрузку шаблонных функций вместо
            их полной специализации.

        \item Полная специализация функций~--- это обычные функции.
    
        \item Виртуальные методы, конструктор по умолчанию, конструктор
            копирования, оператор присваивания и деструктор не могут быть
            шаблонными.
        \item Используйте \code{typedef} или \code{using} для длинных шаблонных имён.
    \end{itemize}
\end{frame}

\end{document}

 
