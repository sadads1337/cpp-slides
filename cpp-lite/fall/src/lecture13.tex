\documentclass{beamer}
\usepackage{csc}
\title{Лекция 13. Сериализация и десериализация}

\date{
   \textbf{ИТМО JB}\\
   14 декабря 2021 \\
   Санкт-Петербург
}

\newcommand{\specialcell}[2][c]{%
  \begin{tabular}[#1]{@{}c@{}}#2\end{tabular}}

\begin{document}
\begin{frame} 
  \titlepage
\end{frame}

\begin{frame}[fragile]{Сериализация и десериализация}
    \begin{itemize}
        \item {\bf Сериализация} ~-- процесс перевода структуры данных в последовательность байтов.
        \item {\bf Десериализация} ~-- создание структуры данных из битовой последовательности (обратная операция).
    \end{itemize}
\end{frame}

\begin{frame}[fragile]{Как можно сериализовать?}
    \begin{itemize}
        \item С помощью текстового представления. \\Например, в форматы: ~{\bf XML}, {\bf JSON}, \dots.
        \item В бинарные форматы. \\Например, в специализированные форматы для хранения больших объемов данных: ~{\bf HDF}, {\bf netCDF} или более старый {\bf GRIB}.
    \end{itemize}
\end{frame}

\begin{frame}[fragile]{Зачем это может быть нужно?}
    \begin{itemize}
        \item Для передачи данных по сетевым протоколам.
        \item Для сохранения информации на файловую систему.
        \item Для удаленного вызова процедур {\bf RPC}.
        \item \dots
    \end{itemize}
\end{frame}

\begin{frame}[fragile]{Как это можно сделать?}
    \begin{minipage}{.5\textwidth}
        {\bf Путь инженера:}
        \begin{itemize}
            \item Выбираем подходящий текстовый формат.
            \item Выбираем подходящую библиотеку для чтения/записи нашего формата.
            \item Реализуем запись/чтение файла с помощью это библиотеки.
        \end{itemize}
    \end{minipage}\hfill
    \begin{minipage}{.5\textwidth}
        {\bf Путь джедая:}
        \begin{itemize}
            \item Придумываем и описываем наш собственный бинарный формат.
            \item Пишем библиотеку/утилиту для работы с нашим форматом.
            \item Реализуем запись/чтение с помощью нашей библиотеки/утилиты.
            \item Чиним баги...
        \end{itemize}
    \end{minipage}
\end{frame}

\begin{frame}[fragile]{Библиотеки для работы с текстовым представлением}
    Для XML:
    \begin{itemize}
        \item pugixml, tinyxml, rapidxml, \dots
    \end{itemize}
    Для JSON:
    \begin{itemize}
        \item jsoncpp, rapidjson, \dots
    \end{itemize}
    Для YML:
    \begin{itemize}
        \item yaml-cpp, rapidyaml, \dots
    \end{itemize}
    Другие форматы: \dots
\end{frame}

\begin{frame}[fragile]{Про путь джедая}
    \begin{center}
        \includegraphics[height=50mm]{dolphin.jpg}
    \end{center}
\end{frame}

\begin{frame}[fragile]{Про путь джедая}
    \begin{center}
        \includegraphics[height=50mm]{enaken.jpg}
    \end{center}
\end{frame}

\begin{frame}[fragile]{Проблемы при разработке формата}
    \begin{itemize}
        \item Высокая сложность в разработке и отладке.
        \item Сложные оптимизация для оптимального хранения данных.
        \item Всегда много багов, требуется хорошее тестовое покрытие.
        \item В перспективе, такой код тяжело поддерживать.
        \item Возникают проблемы с версионированием.
        \item \dots
    \end{itemize}
\end{frame}

\begin{frame}[fragile]{Путь инженера-джедая}
    \begin{itemize}
        \item Для сериализации/десериализации будем использовать готовые и гибкие решения.
        \item Примеров таких решений: {\bf protobuf}, {\bf flatbuffers}, \dots
        \item Позволяют избежать сложностей связанных с разработкой и сконцентрироваться на самом формате (на описании структур данных).
    \end{itemize}
    Далее в качестве примера рассмотрим {\bf protobuf}.
\end{frame}

\begin{frame}[fragile]{Protocol buffers}
    \begin{itemize}
        \item Протокол сериализации (передачи) структурированных данных, 
            предложенный Google как эффективная бинарная альтернатива текстовому формату XML. 
        \item Проще, компактнее и быстрее, чем XML, 
            поскольку осуществляется передача бинарных данных, оптимизированных под минимальный размер сообщения.
        \item Интерфейс поддерживает несколько популярных языков: {\bf C++}, {\bf Java}, {\bf Kotlin}, {\bf C#}, {\bf Go}, \dots
        \item Последняя актуальная версия версия {\bf 3.19.1} (октябрь 2021).
        \item \url{https://developers.google.com/protocol-buffers/docs/cpptutorial}
    \end{itemize}
\end{frame}

\begin{frame}[fragile]{Алгоритм работы с protobuf}
    \begin{itemize}
        \item Описываем наши структуры данных в {\bf proto-файле}.
        \item Компилируем с помощью {\bf protoc}.
        \item Подключаем сгенерированные заголовки к нашему проекту.
        \item Работаем с бинарным представлением. Читаем/записываем/модифицируем.
        \item Возможно расширяем наш {\bf proto-формат}.
    \end{itemize}
\end{frame}

\begin{frame}[fragile]{Описание формата}
    \lstset{basicstyle=\fontsize{7pt}{.8em}\ttfamily,  
    keywordstyle=\fontsize{7pt}{.8em}\color{MOOCBlue}\bfseries, 
    commentstyle=\fontsize{7pt}{.8em}\ttfamily\color{MOOCGreen}}
    \small
    \begin{lstlisting}
syntax = "proto3";

package tutorial;

message Person {
  string name = 1;
  int32 id = 2;
  string email = 3;
  enum PhoneType {
    MOBILE = 0;
    HOME = 1;
    WORK = 2;
  }
  message PhoneNumber {
    string number = 1;
    PhoneType type = 2 [default = HOME];
  }
  repeated PhoneNumber phones = 4;
}

message AddressBook {
  repeated Person people = 1;
}
    \end{lstlisting}
\end{frame}

\begin{frame}[fragile]{Генерация кода}
    Запускаем компиляцию вручную:
\begin{lstlisting}
protoc 
    -I=$SRC_DIR 
    --cpp_out=$DST_DIR 
    $SRC_DIR/addressbook.proto
\end{lstlisting}
\end{frame}

\begin{frame}[fragile]{Генерация кода}
    Используем вспомогательные функции из cmake:
    \lstset{basicstyle=\fontsize{7pt}{.8em}\ttfamily,  
    keywordstyle=\fontsize{7pt}{.8em}\color{MOOCBlue}\bfseries, 
    commentstyle=\fontsize{7pt}{.8em}\ttfamily\color{MOOCGreen}}
    \small
\begin{lstlisting}
include(FindProtobuf)
find_package(Protobuf REQUIRED)
include_directories(${PROTOBUF_INCLUDE_DIR})

# to find *.pb.h files
include_directories(${CMAKE_CURRENT_BINARY_DIR})

protobuf_generate_cpp(PROTO_SRC PROTO_HEADER src/proto/addressbook.proto)
add_library(proto ${PROTO_HEADER} ${PROTO_SRC})
add_executable(main main.cpp)
target_link_libraries(main proto ${PROTOBUF_LIBRARY})
\end{lstlisting}
\end{frame}

\begin{frame}[fragile]{Результат генерации}
    Сгенерировали 2 файла: {\bf addressbook.pb.h} и {\bf addressbook.pb.cc}.
    \lstset{basicstyle=\fontsize{7pt}{.8em}\ttfamily,  
    keywordstyle=\fontsize{7pt}{.8em}\color{MOOCBlue}\bfseries, 
    commentstyle=\fontsize{7pt}{.8em}\ttfamily\color{MOOCGreen}}
    \small
\begin{lstlisting}
// name
inline bool has_name() const;
inline void clear_name();
inline const ::std::string& name() const;
inline void set_name(const ::std::string& value);
inline void set_name(const char* value);
inline ::std::string* mutable_name();

// id
inline bool has_id() const;
inline void clear_id();
inline int32_t id() const;
inline void set_id(int32_t value);

...

\end{lstlisting}
\end{frame}

\begin{frame}[fragile]{Результат генерации}
    Появились дополнительные методы, нужные прежде всего для отладки.
    \lstset{basicstyle=\fontsize{7pt}{.8em}\ttfamily,  
    keywordstyle=\fontsize{7pt}{.8em}\color{MOOCBlue}\bfseries, 
    commentstyle=\fontsize{7pt}{.8em}\ttfamily\color{MOOCGreen}}
    \small
\begin{lstlisting}
// checks if all the required fields have been set.
bool IsInitialized() const;
// returns a human-readable representation of the message, 
// particularly useful for debugging.
string DebugString() const;
// overwrites the message with the given message's values.
void CopyFrom(const Person& from);
// clears all the elements back to the empty state.
void Clear();
\end{lstlisting}
    И методы для чтения/записи:
\begin{lstlisting}
// serializes the message and stores the bytes in the given string.
bool SerializeToString(string* output) const;
// parses a message from the given string.
bool ParseFromString(const string& data);
// writes the message to the given C++ ostream.
bool SerializeToOstream(ostream* output) const;
// parses a message from the given C++ istream.
bool ParseFromIstream(istream* input);
\end{lstlisting}
\end{frame}

\end{document}

 
