\documentclass{beamer}
\usepackage{csc}
\title{Лекция 9. Итераторы. Исключения.}

\date{
   \textbf{ИТМО JB}\\
   2 ноября 2021 \\
   Санкт-Петербург
}

\begin{document}
\begin{frame} 
  \titlepage
\end{frame}

\begin{frame}[fragile]{Категории итераторов}
\small
    \emph{Итератор}~— объект для доступа к элементам
    последовательности, синтаксически похожий на указатель.

    Итераторы делятся на пять категорий: 
    \begin{itemize}
        \item {\bf Random access iterator}: {\tt ++, \verb!--!}, арифметика, {\tt <, >, <=, >=}.\\
        (\texttt{array, vector, deque})
        
        \item \langcpp[20]: {\bf Contiguous iterator}: {\tt ++, \verb!--!}, арифметика, {\tt <, >, <=, >=}. \\
        (\texttt{array, vector})

        \item {\bf Bidirectional iterator}: {\tt ++, \verb!--!}.\\
        (\texttt{list, set, map})

        \item {\bf Forward iterator}: {\tt ++}.\\
        (\texttt{forward\_list, unordered\_set, unordered\_map})

        \item {\bf Input iterator}: {\tt ++}, read-only.

        \item {\bf Output iterator}: {\tt ++}, write-only.
    \end{itemize}
\end{frame}

\begin{frame}[fragile]{Работа с итераторами}
    Функции для работы с итераторами:
\begin{lstlisting}
    void   advance  (Iterator & it, long n);
    size_t distance (Iterator f, Iterator l);
    void   iter_swap(Iterator i, Iterator j);
\end{lstlisting}
\end{frame}

\begin{frame}[fragile]{{\tt iterator\_traits}}
    Позволяет реализовывать алгоритмы в терминах итераторов, даже если тип не имеет соответствующих \code{typedef}.
\begin{lstlisting}
template <class Iterator>
struct iterator_traits {
   typedef difference_type   Iterator::difference_type;
   typedef value_type        Iterator::value_type;
   typedef pointer           Iterator::pointer;
   typedef reference         Iterator::reference;
   typedef iterator_category Iterator::iterator_category;
};
\end{lstlisting}
\end{frame}

\begin{frame}[fragile]{{\tt iterator\_traits}}
\begin{lstlisting}
template <class T>
struct iterator_traits<T *> {
    typedef difference_type   ptrdiff_t;
    typedef value_type        T;
    typedef pointer           T*;
    typedef reference         T&;
    typedef iterator_category random_access_iterator_tag;
};
\end{lstlisting}
\end{frame}

\begin{frame}[fragile]{{\tt iterator\_traits}}
\begin{lstlisting}
template<class BidirIt>
void my_reverse(BidirIt first, BidirIt last)
{
    using diff_t = typename 
        std::iterator_traits<BidirIt>::difference_type;
    diff_t n = std::distance(first, last);
    for (--n; n > 0; n -= 2) {
        uisng value_t = typename 
            std::iterator_traits<BidirIt>::value_type;
        const value_t tmp = *first;
        *first++ = *--last;
        *last = tmp;
    }
}
\end{lstlisting}
\end{frame}
 
\begin{frame}[fragile]{{\tt iterator\_category}}
    Позволяют выбрать более оптимальную реализацию алгоритма.
\begin{lstlisting}
struct random_access_iterator_tag {};
struct bidirectional_iterator_tag {};
struct forward_iterator_tag {};
struct input_iterator_tag {};
struct output_iterator_tag {};
struct contiguous_iterator_tag {}; // C++20
\end{lstlisting}
\end{frame}

\begin{frame}[fragile]{{\tt iterator\_category}}
\begin{lstlisting}
template<class I>
void advance_impl(I & i, long n, random_access_iterator_tag) { 
    i += n; 
}

template<class I>
void advance_impl(I & i, size_t n, ... ) { 
   for (size_t k = 0; k != n; ++k, ++i );
}

template<class I>
void advance(I & i, size_t n) {
    using it_cat = typename 
        iterator_traits<I>::iterator_category
   advance_impl(i, n, it_cat());
}
\end{lstlisting}
\end{frame}

\begin{frame}[fragile]{{\tt reverse\_iterator}}
    У некоторых контейнеров есть обратные итераторы: 
\begin{lstlisting}
list<int> l = { 0, 1, 2, 3, 4, 5, 6, 7, 8, 9 }; 
// list<int>::reverse\_iterator
for(auto i = l.rbegin(); i != l.rend(); ++i)
    cout << *i << endl;
\end{lstlisting}
\end{frame}

\begin{frame}[fragile]{Конвертация итераторов}
Конвертация итераторов:
\begin{lstlisting}
list<int>::iterator i = l.begin();
advance(i, 5); // i указывает на 5
// ri указывает на 4
list<int>::reverse_iterator ri(i); 
i = ri.base();
\end{lstlisting}

Есть возможность сделать обратный итератор из random access\\ или bidirectional при помощи шаблона 
\texttt{reverse\_iterator}.
\begin{lstlisting}
// <iterator>
template <class Iterator> 
class reverse_iterator {...};
\end{lstlisting}
\end{frame}


\begin{frame}[fragile]{Инвалидация итераторов}
    Некоторые операции над контейнерами делают существующие итераторы
    некорректными (\emph{инвалидация} итераторов).
    
    \begin{enumerate}
        \item Удаление делает некорректным итератор на удалённый элемент в любом контейнере.
        \item В \texttt{vector} и \texttt{string}
            добавление потенциально инвалидирует все итераторы (может произойти выделение нового буфера), иначе инвалидируются только итераторы на все следующие элементы.
        \item В \texttt{vector} и \texttt{string} удаление элемента инвалидирует итераторы на все следующие элементы.
        \item В \texttt{deque} удаление/добавление инвалидирует все итераторы, кроме 
            случаев удаления/добавления первого или последнего элементов.
        
    \end{enumerate}
\end{frame}

\begin{frame}[fragile]{Advanced итераторы}
Для пополнения контейнеров:\\
{\tt back\_inserter, front\_inserter, inserter}.
\begin{lstlisting}
// в классе Database
template<class OutIt>
void findByName(string name, OutIt out);
\end{lstlisting}
\begin{lstlisting}
// размер заранее неизвестен
vector<Person> res; 
Database::findByName("Rick", back_inserter(res));
\end{lstlisting}
\end{frame}

\begin{frame}[fragile]{Advanced итераторы}
Для работы с потоками: \\{\tt istream\_iterator, ostream\_iterator}.
\begin{lstlisting}
ifstream file("input.txt");
vector<double> v((istream_iterator<double>(file)), 
                  istream_iterator<double>());
    
copy(v.begin(), v.end(), 
     ostream_iterator<double>(cout, "\n"));
\end{lstlisting}
\end{frame}

\begin{frame}[fragile]{Как написать свой итератор}
\begin{lstlisting}
template <
 class Category, // iterator::iterator\_category
 class T,        // iterator::value\_type
 class Distance = ptrdiff_t,// iterator::difference\_type
 class Pointer = T*,        // iterator::pointer
 class Reference = T&       // iterator::reference
>
struct iterator;
\end{lstlisting}

\begin{lstlisting}
struct PersonIterator 
    : std::iterator<forward_iterator_tag, Person>
{
  // operator++, operator*, ... 
};
\end{lstlisting}
\end{frame}

\begin{frame}[fragile]{Логические ошибки и исключительные ситуации}
\begin{itemize}
\item \textbf{Логические ошибки}.\\
Ошибки в логике работы программы, которые происходят 
из-за неправильно написанного кода, т.е. это ошибки программиста:
\begin{itemize} 
    \item выход за границу массива,
    \item попытка деления на ноль,
    \item обращение по нулевому указателю,
    \item \ldots
\end{itemize}
\item \textbf{Исключительные ситуации}.\\
Ситуации, которые требуют особой обработки.\\
Возникновение таких ситуаций~— это ,,нормальное``  поведение программы.
\begin{itemize} 
    \item ошибка записи на диск,
    \item недоступность сервера,
    \item неправильный формат файла,
    \item \ldots
\end{itemize}
\end{itemize}
\end{frame}

\begin{frame}[fragile]{Выявление логических ошибок на этапе разработки}
    Оператор \code{static\_assert}.\\
    \mbox{}
    \begin{lstlisting}
#include<type_traits>

template<class T> 
void countdown(T start) 
{
    static_assert(std::is_integral<T>::value
               && std::is_signed<T>::value, 
                  "Requires signed integral type");
                  
    while (start >= 0) {
        std::cout << start-- << std::endl;
    }
}
    \end{lstlisting}
\end{frame}

\begin{frame}[fragile]{Выявление логических ошибок на этапе разработки}
    Макрос \code{assert}.
    \begin{lstlisting}
#include<type_traits>
//\#define NDEBUG
#include <cassert>

template<class T> 
void countdown(T start) 
{
    assert(start >= 0);
    while (start >= 0) {
        std::cout << start-- << std::endl;
    }
}
    \end{lstlisting}
\end{frame}

\begin{frame}[fragile]{Способы сообщения об ошибке}
\small
        \fakeitem Возврат статуса операции:
    \begin{lstlisting}
bool write(string file, string data, size_t & bytes);
    \end{lstlisting}

        \fakeitem Возврат кода ошибки:
    \begin{lstlisting}
int const OK = 0, IO_WRITE_FAIL = 1, IO_OPEN_FAIL = 2;
int write(string file, string data, size_t & bytes);
    \end{lstlisting}

        \fakeitem Глобальная переменная для кода ошибки:
    \begin{lstlisting}
size_t write(string file, string data);
    \end{lstlisting}

    \begin{lstlisting}
size_t bytes = write(f, data);
if (errno) {
    cerr << strerror(errno);
    errno = 0;
}
\end{lstlisting}
\end{frame}

\begin{frame}[fragile]{Исключения}
\begin{lstlisting}
size_t write(string file, string data) {
    if (!open(file)) throw FileOpenError(file);
    //...
}

double safediv(int x, int y) {
    if (y == 0) throw MathError("Division by zero");
    return double(x) / y;
}
\end{lstlisting}
\end{frame}

\begin{frame}[fragile]{Исключения}
\begin{lstlisting}
void write_x_div_y(string file, int x, int y) {
    try { 
        write(file, to_string(safediv(x, y)));
    } catch (MathError & s) { 
        // обработка ошибки в safediv
    } catch (FileError & e) { 
        // обработка ошибки в write
    } catch (...) {
        // все остальные ошибки
    }
}
\end{lstlisting}
\end{frame}

\begin{frame}[fragile]{Stack unwinding}
При возникновении исключения объекты на стеке
уничтожаются в естественном (обратном) порядке.
\begin{lstlisting}
void foo() {
    A a;
    if (!a) throw Error();
    B b;
}
void bar() {
    C d;
    try {
        D d;
        foo();
    } catch (const Error &) { throw OtherError; }
}
\end{lstlisting}
\end{frame}

\begin{frame}[fragile]{Почему не стоит бросать встроенные типы}
\begin{lstlisting}
int foo() {
    if (...) throw -1;
    if (...) throw 3.1415;
}
void bar(int a) {
    if (a == 0) throw string("Not my fault!");
}
int main () {
    try { bar(foo()); }
    catch (string & s) { /*only str*/ }
    catch (int a) { /*only int*/ }
    catch (double d) { /*only double*/ }
    catch (...) { /*nothing*/ }
}
\end{lstlisting}
\end{frame}

\begin{frame}[fragile]{Стандартные классы исключений}
    Базовый класс для всех исключений (в {\tt <exception>}):
    \begin{lstlisting}
struct exception {
  virtual ~exception();
  virtual const char* what() const;
}; 
    \end{lstlisting}

    Стандартные классы ошибок (в {\tt stdexcept}):\\
    \begin{itemize}
        \item {\tt logic\_error}: 
     {\tt domain\_error}, {\tt invalid\_argument}, {\tt length\_error}, {\tt
     out\_of\_range} 
        \item {\tt runtime\_error}: 
        {\tt range\_error},
        {\tt overflow\_error}, 
        {\tt underflow\_error}
    \end{itemize}
\end{frame}

 

\begin{frame}[fragile]{Исключения в стандартной библиотеке}
\small
\begin{itemize}
    \item Метод \texttt{at} контейнеров {\tt array}, {\tt vector}, {\tt deque}, {\tt basic\_string}, {\tt bitset}, {\tt map}, {\tt unordered\_map} бросает {\tt out\_of\_range}.
    \item Оператор \code{new} {\tt T} бросает {\tt bad\_alloc}.\\
    Оператор {\tt \code{new} (std::nothrow) T} в возвращает 0.
    \item Оператор \code{typeid} от разыменованного нулевого указателя бросает {\tt bad\_typeid}.

\end{itemize}
\end{frame}

\begin{frame}[fragile]{Исключения в стандартной библиотеке}
\small
\begin{itemize}
    \item Потоки ввода-вывода. 
    \begin{lstlisting}
std::ifstream file;
file.exceptions(std::ifstream::failbit
    | std::ifstream::badbit);
try {
    file.open("test.txt");
    cout << file.get() << endl;
    file.close();
}
catch (std::ifstream::failure const& e) {
    cerr << e.what() << endl;
}
    \end{lstlisting}
\end{itemize}
\end{frame}


\begin{frame}[fragile]{Исключения в деструкторах}
Исключения не должны покидать деструкторы.
\begin{itemize}
\item Двойное исключение:
\begin{lstlisting}
void foo() {
    try {
        Bad b; // исключение в деструкторе
        bar(); // исключение
    } catch (std::exception & e) {
        // ...
    }
}
\end{lstlisting}
\item Неопределённое поведение:
\begin{lstlisting}
void bar() {
    Bad * bad = new Bad[100];
    delete [] bad; // ислючение в деструкторе
}
\end{lstlisting}
\end{itemize}
\end{frame}

\begin{frame}[fragile]{Исключения в конструкторе}
Исключения — это единственный способ
прервать  конструирование объекта и сообщить об ошибке.

\begin{lstlisting}
struct Database {
   explicit Database(string const& uri) {
      if (!connect(uri)) throw ConnectionError(uri);
   }
   ~Database() { disconnect(); }
};
int main() try {
    Database * db = new Database("db.local");
    db->dump("db-local-dump.sql");
    delete db;
} catch (std::exception const& e) {
    std::cerr << e.what() << std::endl;
}

\end{lstlisting}
\end{frame}

\begin{frame}[fragile]{Исключения в списке инициализации}
Позволяет отловить исключения при создании полей класса.
\begin{lstlisting}
struct System 
{
    System(string const& uri, string const& data)
    try : db_(uri), dh_(data)
    { ... }
    catch (std::exception & e) 
    {
        log("System constructor: ", e);
        throw;
    }
    
    Database    db_;
    DataHolder  dh_;
};   
\end{lstlisting}
\end{frame}


\begin{frame}[fragile]{Как обрабатывать ошибки?}
    Есть несколько ,,правил хорошего тона``.
    \begin{itemize}
        \item Разделяйте ошибки программиста и исключительные ситуации.
        \item Используйте \code{assert} и \code{static\_assert} для выявления ошибок на этапе разработки.
        \item В пределах одной логической части кода обрабатывайте ошибки централизованно и однообразно.
        \item Обрабатывайте ошибки там, где их можно обработать.
        \item Если в данном месте ошибку не обработать, то пересылайте её выше при помощи исключения.
        \item Бросайте только стандартные классы исключений или производные от них.
        \item Бросайте исключения по значению, а ловите по ссылке.
        \item Отлавливайте все исключения в точке входа.

    \end{itemize}
\end{frame}

\end{document}

 
