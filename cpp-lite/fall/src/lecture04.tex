\documentclass{beamer}
\usepackage{csc}
\title{Лекция 4. Ручное управление памятью}

\date{
   \textbf{CS центр}\\
   28 сентября 2021 \\
   Санкт-Петербург
}

\begin{document}
\begin{frame} 
  \titlepage
\end{frame}

\begin{frame}[fragile]{Еще раз про указатели}
    \begin{itemize}
        \item Указатель — это переменная, хранящая адрес некоторой 
            ячейки памяти.
        \item Указатели являются типизированными.
\begin{lstlisting}
int i = 3; // переменная типа int
int * p = 0; // указатель на переменную типа int
\end{lstlisting}
        \item Нулевому указателю (\code{nullptr}) не~соответствует никакая ячейка памяти.
        \item Оператор взятия адреса переменной \code{\&}.
        \item Оператор разыменования \code{*}.

\begin{lstlisting}
p = &i; // указатель p указывает на переменную i
*p = 10; // изменяется ячейка по адресу p, т.е. i
\end{lstlisting}
    \end{itemize}
\end{frame}

\begin{frame}[fragile]{Указатели и const}
    \begin{itemize}
        \item Константный указатель~--- это переменная, хранящая адрес некоторой 
            ячейки памяти, который нельзя изменить.
        \item Указатель на константу~--- это переменная, хранящая адрес некоторой 
            ячейки памяти. Данные в этой ячейки нельзя изменить.
        \item Модификатор~--- часть типа.
\begin{lstlisting}
int i = 3;
int * const p1 = &i; // константный указатель
int const * p2 = &i; // указатель на константу
const int * p3 = &i; // указатель на константу

p1 = nullptr; // error
*p1 = 10; // ok
p2 = nullptr; // ok
*p2 = 10; // error
\end{lstlisting}
    \end{itemize}
\end{frame}

\begin{frame}[fragile]{Связь массивов и указателей}
    \begin{itemize}
        \item Указатели позволяют передвигаться по массивам.
        \item Для этого используется арифметика указателей:
\begin{lstlisting}
int m[10] = {1, 2, 3, 4, 5}; 
int * p = &m[0]; // адрес начала массива
int * q = &m[9]; // адрес последнего элемента
\end{lstlisting}
            \begin{itemize}
                \item {\tt (p + k)} — сдвиг на {\tt k} ячеек типа \code{int} вправо.
                \item {\tt (p - k)} — сдвиг на {\tt k} ячеек типа \code{int} влево.
                \item {\tt (q - p)} — количество ячеек между
                    указателями.
                \item {\tt p[k]} эквивалентно {\tt *(p + k)}.
            \end{itemize}
    \end{itemize}
\begin{center}
\begin{tikzpicture}[
      start chain=1 going right,node distance=-0.15mm
    ]
    \tikzstyle{cell} = [rectangle,minimum width=5mm, minimum height=5mm,draw,on chain=1];
    \tikzstyle{arrow} = [>=stealth',shorten >=2pt, shorten <=2pt]

    \foreach \i in {1,2}
        \node[cell,dashed] {};
        
    \node[cell,dashed] (b) {};

    \foreach \i in {1,2,3,4,5}
        \node[cell,thick] (m\i) {\i};
        
    \foreach \i in {0,0,0}
        \node[cell,thick] {\i};

    \node[cell,thick] (e) {0};
    \node[cell,thick] {0};

    \node[below of=b,yshift=-10mm] (p) {{\tt p}};
    \path [arrow,->] (p) edge node {} (b.south east);

    \node[below of=e,yshift=-10mm] (q) {{\tt q}};
    \path [arrow,->] (q) edge node {} (e.south east);

    \node[below of=m2,yshift=-10mm] (p2) {{\tt (p + 2)}};
    \path [arrow,->] (p2) edge node {} (m2.south east);

    \node[below of=m5,yshift=-10mm] (q4) {{\tt (q - 4)}};
    \path [arrow,->] (q4) edge node {} (m5.south east);

    \foreach \i in {1,2,3}
        \node[cell,dashed] {};
\end{tikzpicture}
\end{center}
\end{frame}

\begin{frame}[fragile]{Примеры}
Заполнение массива:
\begin{lstlisting}
int m[10] = {}; // изначально заполнен нулями

for (int * p = m ; p <= m + 9; ++p) {
    *p = (p - m) + 1;
} // Массив заполнен числами от 1 до 10
\end{lstlisting}
Передача массива в функцию:
\begin{lstlisting} 
int max_element(int * m, int size) {
    int max = *m; 
    for (int i = 1; i < size; ++i) {
        if (m[i] > max) { max = m[i]; }
    }
    return max;
}
\end{lstlisting}
\end{frame}


\begin{frame}[fragile]{Два способа передачи массива}
\small
\begin{lstlisting}
bool contains(int * m, int size, int value) {
    for (int i = 0; i != size; ++i) {
        if (m[i] == value) { return true; }
    }
    return false;
}
bool contains(int * p, int * q, int value) {
    for (; p != q; ++p) {
        if (*p == value) { return true; }
    }
    return false;
}
\end{lstlisting}
\begin{center}
\begin{tikzpicture}[
      start chain=1 going right,node distance=-0.15mm
    ]
    \tikzstyle{cell} = [rectangle,minimum width=5mm, minimum height=5mm,draw,on chain=1];
    \tikzstyle{arrow} = [>=stealth',shorten >=2pt, shorten <=2pt]

    \foreach \i in {1,2}
        \node[cell,dashed] {};
        
    \node[cell,dashed] (b) {};

    \foreach \i in {1,2,3,4,5}
        \node[cell,thick] (m\i) {\i};
        
    \foreach \i in {0,0,0}
        \node[cell,thick] {\i};

    \node[cell,thick] {0};
    \node[cell,thick] (e) {0};

    \node[below of=b,yshift=-10mm] (p) {{\tt p}};
    \path [arrow,->] (p) edge node {} (b.south east);

    \node[below of=e,yshift=-10mm] (q) {{\tt q}};
    \path [arrow,->] (q) edge node {} (e.south east);

    \node[cell,dashed,color=red] {};

    \foreach \i in {2,3}
        \node[cell,dashed] {};
\end{tikzpicture}
\end{center}
\end{frame}

\begin{frame}[fragile]{Возрат указателя из функции}
Функция для поиска максимума в массиве:
\begin{lstlisting}
int * max_element(int * p, int * q) {
    int * pmax = p;
    for (; p != q; ++p) {
        if (*p > *pmax) { pmax = p; }
    }
    return pmax;
}
\end{lstlisting}
\begin{lstlisting}
    int m[10] = {...};
    int * pmax = max_element(m, m + 10); 
    cout << "Maximum = " << *pmax << endl; 
\end{lstlisting}
\end{frame}

\begin{frame}[fragile]{Возрат значения через указатель}
Функция для поиска максимума в массиве:
\begin{lstlisting}
bool max_element(int * p, int * q, int * res) {
    if (p == q) { return false; }
    *res = *p;
    for (; p != q; ++p) {
        if (*p > *res) { *res = *p; }
    }
    return true;
}
\end{lstlisting}
\begin{lstlisting}
    int m[10] = {...};
    int max = 0;
    if (max_element(m, m + 10, &max)) {
        cout << "Maximum = " << max << endl;
    }
\end{lstlisting}
\end{frame}

\begin{frame}[fragile]{Возрат значения через указатель на указатель}
Функция для поиска максимума в массиве:
\begin{lstlisting}
bool max_element(int * p, int * q, int ** res) {
    if (p == q)
        return false;
    *res = p;
    for (; p != q; ++p)
        if (*p > **res)
            *res = p;
    return true;
}
\end{lstlisting}
\begin{lstlisting}
    int m[10] = {...};
    int * pmax = 0;
    if (max_element(m, m + 10, &pmax)) 
        cout << "Maximum = " << *pmax << endl; 
\end{lstlisting}
\end{frame}

\begin{frame}[fragile]{Недостатки указателей}
    \begin{itemize}
        \item Использование указателей синтаксически загрязняет код
            и усложняет его понимание.
            (Приходится использовать операторы \code{*} и \code{\&}.)
        \item Указатели могут быть неинициализированными (некорректный код).
        \item Указатель может быть нулевым (корректный код), 
            а значит указатель нужно проверять на равенство нулю.
        \item Арифметика указателей может сделать из корректного указателя
            некорректный (легко промахнуться).
        \item Тяжелее отследить время жизни указателей.
    \end{itemize}
\end{frame}

\begin{frame}[fragile]{Еще раз про различия ссылок и указателей}
    \begin{itemize}
        \item Ссылка не может быть неинициализированной.
\begin{lstlisting}
int * p; // OK
int & l; // ошибка
\end{lstlisting}
        \item У ссылки нет нулевого значения.            
\begin{lstlisting}
int * p = 0; // OK
int & l = 0; // ошибка
\end{lstlisting}
    \item Ссылку нельзя переприсвоить:
\begin{lstlisting}
int a = 10, b = 20;
int * p = &a; // p указывает на a
p = &b;       // p указывает на b
int & l = a;  // l ссылается на a
l = b;        // a присваивается значение b
\end{lstlisting}
\end{itemize}
\end{frame}

\begin{frame}[fragile]{Еще раз про различия ссылок и указателей}
    \begin{itemize}
        \item Нельзя получить адрес ссылки или ссылку на ссылку.
\begin{lstlisting}
int a = 10;
int * p = &a;  // p указывает на a
int ** pp = &p;// pp указывает на переменную p
int & l = a;   // l ссылается на a
int * pl = &l; // pl указывает на переменную a
int && ll = l; // ошибка
\end{lstlisting}
        \item Нельзя создавать массивы ссылок.
\begin{lstlisting}
int * mp[10] = {};  // массив указателей на int
int & ml[10] = {};  // ошибка
\end{lstlisting}
        \item Для ссылок нет арифметики.
    \end{itemize}
\end{frame}

\begin{frame}[fragile]{Ссылки и rvalue}
    Cсылки могут указывать только на lvalue.
    \begin{lstlisting}
int a = 10, b = 20;
int m[10] = {1,2,3,4,5,5,4,3,2,1};
int & l1 = a;       // OK
int & l2 = a + b;   // ошибка
int & l3 = *(m + a / 2);        // OK
int & l4 = *(m + a / 2) + 1;    // ошибка
int & l5 = (a + b > 10) ? a : b;  // OK
    \end{lstlisting}
\end{frame}

\begin{frame}[fragile]{Зачем нужна динамическая память?}
    \begin{itemize}
        \item Стек программы ограничен. Он не предназначен
            для хранения больших объемов данных.
\begin{lstlisting}
// Не умещается на стек
double m[10000000] = {}; // 80 Mb
\end{lstlisting}
        \item Время жизни локальных переменных ограничено
            временем работы функции.
        \item Динамическая память выделяется в сегменте данных.
        \item Структура, отвечающая за выделение дополнительной
            памяти, называется {\bf кучей} (не нужно путать
            с одноимённой структурой данных).
        \item Выделение и освобождение памяти {\em управляется вручную}.
    \end{itemize}
\end{frame}

\begin{frame}[fragile]{Выделение памяти в стиле \langc}
    \begin{itemize}
        \item Стандартная библиотека \code{cstdlib}
            предоставляет четыре функции для управления памятью:
            {\small
            \begin{lstlisting}
void * malloc(size_t size);
void * calloc(size_t nmemb, size_t size);
void * realloc(void * ptr, size_t size);
void free(void * ptr);
            \end{lstlisting}}
        \item \code{size\_t}~--- специальный целочисленный беззнаковый тип, 
            может вместить в себя размер любого типа в байтах.
        \item Тип \code{size\_t} используется для указания размеров типов данных,
            для индексации массивов и пр.
        \item \code{void *}~--- это указатель на нетипизированную память\\
            (раньше для этого использовалось \code{char *}).
    \end{itemize}
\end{frame}

\begin{frame}[fragile]{Выделение памяти в стиле \langc}
    \begin{itemize}
        \item Функции для управления памятью в стиле \langc:
            {\small
\begin{lstlisting}
void * malloc(size_t size);
void * calloc(size_t nmemb, size_t size);
void * realloc(void * ptr, size_t size);
void free(void * ptr);               
\end{lstlisting}
}
        \item \code{malloc}~--- выделяет область памяти размера $\ge$
            \code{size}. Данные не инициализируются.
        \item \code{calloc}~--- выделяет массив из \code{nmemb} элементов  размера 
            \code{size}. Данные инициализируются нулём.
        \item \code{realloc}~--- изменяет размер области памяти по указателю
            \code{ptr} на \code{size} (если возможно, то это делается на месте).
        \item \code{free}~--- освобождает область памяти, ранее выделенную
            одной из функций \code{malloc}/\code{calloc}/\code{realloc}.
    \end{itemize}
\end{frame}

\begin{frame}[fragile]{Выделение памяти в стиле \langc}
    \begin{itemize}
        \item Для указания размера типа используется оператор \code{sizeof}.
\begin{lstlisting}
// создание массива из 1000 int    
int * m = (int *)malloc(1000 * sizeof(int));
m[10] = 10;

// изменение размера массива до 2000
m = (int *)realloc(m, 2000 * sizeof(int));

// освобождение массива
free(m);

// создание массива нулей
m = (int *)calloc(3000, sizeof(int));
free(m);
m = 0;
\end{lstlisting}
    \end{itemize}
\end{frame}

\begin{frame}[fragile]{Выделение памяти в стиле \langcpp}
    \begin{itemize}
        \item Язык \langcpp предоставляет два набора операторов для
            выделения памяти:
            \begin{enumerate}
                \item \code{new} и \code{delete}~--- для одиночных значений,

                \item \code{new []} и \code{delete []}~--- для массивов.
            \end{enumerate}

        \item Версия оператора \code{delete} должна соответствовать
            версии оператора \code{new}.
\begin{lstlisting}
// выделение памяти под один int со значением 5
int * m = new int(5);
delete m; // освобождение памяти

// создание массива нулей
m = new int[1000](); // () означает обнуление 
delete [] m; // освобождение памяти
\end{lstlisting}
    \end{itemize}
\end{frame}

\begin{frame}[fragile]{Типичные проблемы при работе с памятью}
    \begin{itemize}
        \item Проблемы производительности: создание переменной на стеке
            намного ``дешевле'' выделения для неё динамической памяти.

        \item Проблема фрагментации: выделение большого количества
            небольших сегментов способствует фрагментации памяти.
        
        \item Утечки памяти:
\begin{lstlisting}
// создание массива из 1000 int    
int * m = new int[1000];

// создание массива из 2000 int    
m = new int[2000]; // утечка памяти

// Не вызван delete [] m, утечка памяти
\end{lstlisting}
    \end{itemize}
\end{frame}

\begin{frame}[fragile]{Типичные проблемы при работе с памятью}
    \begin{itemize}
    \item Неправильное освобождение памяти.
\begin{lstlisting}
int * m1 = new int[1000];
delete m1; // должно быть delete [] m1

int * p = new int(0);
free(p); // совмещение функций C++ и C

int * q1 = (int *)malloc(sizeof(int));
free(q1);
free(q1); // двойное удаление

int * q2 = (int *)malloc(sizeof(int));
free(q2);
q2 = 0;   // обнуляем указатель
free(q2); // правильно работает для q2 = 0
\end{lstlisting}
\end{itemize}
\end{frame}

\begin{frame}[fragile]{Многомерные встроенные массивы}
    \begin{itemize}
        \item \langcpp позволяет определять многомерные массивы:
        \begin{lstlisting}
int m2d[2][3] = { {1, 2, 3}, {4, 5, 6} };
for( size_t i = 0; i != 2; ++i ) {
    for( size_t j = 0; j != 3; ++j ) {
        cout << m2d[i][j] << ' ';
    }
    cout << endl;
}
        \end{lstlisting}
    \item Элементы {\tt m2d} располагаются в памяти ``по строчкам''.
    \item Размерность массивов может быть любой, но на практике
        редко используют массивы размерности $> 4$.
\begin{lstlisting}
int m4d[2][3][4][5] = {};
\end{lstlisting}
    \end{itemize}
\end{frame}

\begin{frame}[fragile]{Динамические массивы}
    \begin{itemize}
        \item Для выделения одномерных динамических массивов
            обычно используется оператор \code{new []}.
\begin{lstlisting}
int * m1d = new int[100];
\end{lstlisting}
        \item Какой тип должен быть у указателя на двумерный динамический
            массив?
            \begin{itemize}
                \item Пусть {\tt m}~--- указатель на двумерный массив типа \code{int}.

                \item Значит {\tt m[i][j]} имеет тип \code{int} (точнее
                    \code{int \&}).

                \item {\tt m[i][j] $\Leftrightarrow$ *(m[i] + j)},
                    т.е. тип {\tt m[i]}~--- \code{int *}.

                \item аналогично, {\tt m[i] $\Leftrightarrow$ *(m + i)},
                    т.е. тип {\tt m}~--- \code{int **}.
            \end{itemize}
            \item Чему соответствует значение {\tt m[i]}?\\ 
                Это адрес строки с номером {\tt i}.

            \item Чему соответствует значение {\tt m}?\\
                Это адрес массива с указателями на строки.
    \end{itemize}
\end{frame}

\begin{frame}[fragile]{Двумерные массивы}
    Давайте рассмотрим создание массива $5\times 4$.
    
\begin{center}
\begin{tikzpicture}[                                                          
      start chain=1 going right, node distance=-0.15mm,
      start chain=2 going right
    ]
    \tikzstyle{cell} = [rectangle,minimum width=4mm, minimum height=4mm,draw];
    \tikzstyle{arrow} = [>=stealth']

    \begin{scope}[shift={(4cm,35mm)}]
    \foreach \i in {1,2,3,4} {
        \node[cell,on chain=2] (m\i-1) {};   
            \foreach \j/\k in {{2/1},{3/2},{4/3},{5/4}}
                \node[cell,below of=m\i-\k,yshift=-7mm] (m\i-\j) {};
    };
   
    \node[on chain=2] (t0) {{\tt m[0]}};   
    \foreach \i/\k in {{1/0},{2/1},{3/2},{4/3}}
        \node[below of=t\k,yshift=-7mm] (t\i) {{\tt m[\i]}};
    \end{scope}

    \node[on chain=1] (m) {{\tt m}};
    \foreach \i in {1,2,3,4,5} {
        \node[cell,on chain=1] (m\i) {};
        \path [arrow,->] (m\i.center) edge node {} (m1-\i.west);     
    };
\end{tikzpicture}
        \begin{lstlisting}
int ** m = new int * [5];
for (size_t i = 0; i != 5; ++i)
    m[i] = new int[4];
        \end{lstlisting}
\end{center}
\end{frame}

\begin{frame}[fragile]{Двумерные массивы}
    Выделение и освобождение двумерного массива размера $a\times b$.
        \begin{lstlisting}
int ** create_array2d(size_t a, size_t b) {
    int ** m = new int *[a];
    for (size_t i = 0; i != a; ++i)
        m[i] = new int[b];
    return m;
}

void free_array2d(int ** m, size_t a, size_t b) {
    for (size_t i = 0; i != a; ++i)
        delete [] m[i];
    delete [] m;
}
        \end{lstlisting}
При создании массива оператор \code{new} вызывается $(a + 1)$ раз.
\end{frame}

\begin{frame}[fragile]{Двумерные массивы: эффективная схема}
    Рассмотрим эффективное создание массива $5\times 4$.
    
\begin{center}
\begin{tikzpicture}[                                                          
      start chain=1 going right, node distance=-0.15mm,
      start chain=2 going right
    ]
    \tikzstyle{cell} = [rectangle,minimum width=4mm, minimum height=4mm,draw];
    \tikzstyle{arrow} = [>=stealth']
    \tikzstyle{mbrace} = [decorate,decoration={brace,amplitude=5pt}];

    \begin{scope}[shift={(0cm,25mm)}]
    \foreach \i in {0,1,2,3,4} {
        \node[cell,on chain=2] (m\i-1) {};
        \node[above of=m\i-1, shift={(6mm,7mm)}] (t\i) {{\tt\small m[\i]}};
         \foreach \j in {2,3,4}
             \node[cell,on chain=2] (m\i-\j) {};
        \draw[mbrace] 
            (m\i-1.north west)--(m\i-4.north east) node[midway,anchor=west,xshift=6pt] {};
    };
    \end{scope}

    \node[on chain=1] (m) {{\tt m}};
    \foreach \i in {0,1,2,3,4} {
        \node[cell,on chain=1] (m\i) {};
        \path [arrow,->] (m\i.center) edge node {} (m\i-1.south west);     
    };
\end{tikzpicture}
        \begin{lstlisting}
int ** m = new int * [5];
m[0] = new int[5 * 4];
for (size_t i = 1; i != 5; ++i)
    m[i] = m[i - 1] + 4;
        \end{lstlisting}
\end{center}
\end{frame}

\begin{frame}[fragile]{Двумерные массивы: эффективная схема}
    Эффективное выделение и освобождение двумерного массива размера $a\times b$.

    \begin{lstlisting}
int ** create_array2d(size_t a, size_t b) {
    int ** m = new int *[a];
    m[0] = new int[a * b];
    for (size_t i = 1; i != a; ++i)
        m[i] = m[i - 1] + b;
    return m;
}

void free_array2d(int ** m, size_t a, size_t b) {
    delete [] m[0];
    delete [] m;
}
        \end{lstlisting}
При создании массива оператор \code{new} вызывается 2 раза.
\end{frame}

\begin{frame}[fragile]{Умные указатели}
    \begin{enumerate}
        \item Идиома RAII (Resource Acquisition Is Initialization):
            время жизни ресурса связанно с временем жизни объекта.
            \begin{itemize}
                \item Получение ресурса в конструкторе.
                \item Освобождение ресурса в деструкторе.
            \end{itemize}

        \item Основные области использования RAII:
            \begin{itemize}
                \item для управления памятью,
                \item для открытия файлов или устройств,
                \item для критических секций при параллельном исполнении кода.
            \end{itemize}
        \item Умные указатели~--- объекты, инкапсулирующие владение памятью.
            Синтаксически ведут себя так же, как и обычные указатели.
    \end{enumerate}
\end{frame}

\begin{frame}[fragile]{Основные стратегии}
    \begin{enumerate}
        \item \code{scoped\_ptr}~--- время жизни объекта ограничено временем
            жизни умного указателя.
        \item \code{shared\_ptr}~--- разделяемый объект, реализация с подсчётом
            ссылок.
        \item \code{weak\_ptr}~--- разделяемый объект, реализация с подсчётом
            ссылок, слабая ссылка (используется вместе с \code{shared\_ptr}).
        \item \code{unique\_ptr}~--- эксклюзивное владение
            объектом с передачей владения при перемещении. 
        \item \code{intrusive\_ptr}~--- разделяемый объект, реализация самим
            внутри объекта.
        \item \code{linked\_ptr}~--- разделяемый объект, реализация списком
            указателей.
    \end{enumerate}
\end{frame}

\begin{frame}[fragile]{\texttt{unique\_ptr}}
    \small
    \begin{itemize}
        \item Определен в заголовочном файле \code{<memory>}.
        \item Для передачи и возврата указателей из функции.
        \item Владение эксклюзивно и передаётся при только при перемещении.
    \end{itemize}
        \begin{lstlisting}
void foo(const std::unique_ptr<int> & ptr) {
    std::cout << *ptr;
}
void bar(std::unique_ptr<int> && ptr) {
    std::cout << *ptr;
}
void baz(std::unique_ptr<int> ptr) {
    std::cout << *ptr;
}
        \end{lstlisting}
\end{frame}

\begin{frame}[fragile]{\texttt{unique\_ptr}}
    \begin{lstlisting}

int main() {
    std::unique_ptr<int> ptr = std::make_unique<int>(10);

    auto & ptr_ref = ptr; // ok
    auto ptr_copy = ptr; // error

    foo(ptr); // ok
    bar(ptr); // error
    bar(std::move(ptr)); // ok
    baz(ptr); // error
    baz(std::move(ptr)); // ok
}
    \end{lstlisting}
\end{frame}


\begin{frame}[fragile]{\texttt{shared\_ptr}}
    \small
    \begin{itemize}
        \item Для разделяемых объектов.
        \item Ведётся подсчёт ссылок.
        \item Нельзя вернуть владение объектом.
    \end{itemize}
        \begin{lstlisting}
void foo(const std::shared_ptr<int> & ptr) {
    std::cout << *ptr;
}
void bar(std::shared_ptr<int> && ptr) {
    std::cout << *ptr;
}
void baz(std::shared_ptr<int> ptr) {
    std::cout << *ptr;
}
        \end{lstlisting}
\end{frame}

\begin{frame}[fragile]{\texttt{shared\_ptr}}
    \begin{lstlisting}

int main() {
    std::shared_ptr<int> ptr = std::make_shared<int>(10);

    auto & ptr_ref = ptr; // no increment
    auto ptr_copy = ptr; // increment

    foo(ptr); // no increment
    bar(ptr); // error
    bar(std::move(ptr)); // no increment
    baz(ptr) // increment
    baz(std::move(ptr)); // no increment
}
    \end{lstlisting}
\end{frame}

\begin{frame}[fragile]{\texttt{shared\_ptr}}
    \begin{lstlisting}
void foo() {
    // counter == 1
    std::shared_ptr<int> ptr = std::make_shared<int>(10);

    {
        // counter == 2
        auto ptr_copy = ptr;
    } // counter == 1

    {
        // counter == 1
        auto & ptr_ref = ptr;
    } // counter == 1
} // counter == 0
    \end{lstlisting}
\end{frame}

\begin{frame}[fragile]{\texttt{shared\_ptr}}
    \begin{lstlisting}
void foo() {
    // counter == 1
    std::shared_ptr<int> ptr = std::make_shared<int>(10);

    {
        // counter == 1
        auto ptr_other = std::move(ptr);
    } // counter == 0
}
    \end{lstlisting}
\end{frame}

\begin{frame}[fragile]{\texttt{weak\_ptr}}
    \begin{itemize}
        \item Для использования вместе с \texttt{shared\_ptr}.
        \item Слабая ссылка для исключения циклических зависимостей.
        \item Не владеет объектом.
    \end{itemize}

    \begin{lstlisting}
void foo(std::weak_ptr & ptr_weak) {
    if (auto ptr_locked = ptr_weak.lock()) {
        // shared here
    }
}

int main() {
    std::shared_ptr<int> ptr = std::make_shared<int>(10);
    foo(ptr); // implicit cast here
}
    \end{lstlisting}
\end{frame}

\begin{frame}[fragile]{Умные указатели и const}
    \begin{itemize}
\begin{lstlisting}
// константный указатель
const auto p1 = std::make_shared<int>(10);
// указатель на константу
auto p2 = std::make_shared<const int>(10);

p1 = nullptr; // error
*p1 = 10; // ok
p2 = nullptr; // ok
*p2 = 10; // error
\end{lstlisting}
    \end{itemize}
\end{frame}

\begin{frame}[fragile]{Заключение}
\small
\begin{itemize}
    \item Умные указатели намного удобнее ручного управления памятью.
    \item Для локальных объектов~--- \code{scoped\_ptr} (подробнее на семинаре).
    \item Для разделяемых объектов~--- \code{shared\_ptr}.
    \item В сильносвязанных системах рассмотрите возможность использовать \code{weak\_ptr}.    
    \item Используйте \code{intrusive\_ptr} для тех объектов, которые
        сами управляют своим временем жизни.
    \item Прочитайте документацию по \code{shared\_ptr} и \code{unique\_ptr}.
\end{itemize}
\end{frame}

\end{document}
