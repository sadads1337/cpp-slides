\documentclass{beamer}
\usepackage{csc}
\title{Лекция 2. Переменные, ссылки, массивы}

\date{
   \textbf{ИТМО JB}\\
   14 сентября 2021 \\
   Санкт-Петербург
}

\begin{document}
\begin{frame} 
  \titlepage
\end{frame}

\begin{frame}[fragile]{Еще раз о переменных}
    \begin{itemize}
        \item Имя переменной не должно начинаться с цифры.
        \item Имя переменной не должно включать следующие символы: \\
        \begin{enumerate}
            \item {\tt ,} {\tt /}, {\tt :},
            \item {\tt *}, {\tt ?}, {\tt .},
            \item {\tt <}, {\tt >}, {\tt |},
            \item \dots
        \end{enumerate}
        \item Имена, начинающиеся на \code{\_\_}, зарезирвированны стандартом. Например, \code{\_\_func\_\_}
    \end{itemize}
\end{frame}

\begin{frame}[fragile]{Как реализовать swap}
    Рассмотрим функцию, меняющую параметры местами:
    \begin{lstlisting}
void swap(int a, int b) {
    int t = a;
    a = b;
    b = t;
}

int main() {
    int k = 10, m = 20;
    swap(k, m);
    std::cout << k << ' ' << m << std::endl; // 10 20
    return 0;
}
    \end{lstlisting}
    Проблема: {\tt swap} изменяет локальные копии переменных {\tt k} и {\tt m}.
\end{frame}

\begin{frame}[fragile]{Ссылки}
    \begin{itemize}
        \item {\bf Ссылка} — это некоторый ``синоним’’ для имени переменной.
        \item Ссылки бывают 2 типов: {\bf константная} и {\bf неконстантная}. Например:
            \begin{lstlisting}
int i = 3; // переменная типа int
int & i_ref = i; // ссылка на i
const int & i_cref = i; // константная ссылка на i
            \end{lstlisting}
        \item Аналогичные записи с некоторым ``синтаксическим сахаром’’:
            \begin{lstlisting}
auto & i_ref = i;
const auto & i_cref = i;
int & const i_cref2 = i;
auto & const i_cref2 = i;
            \end{lstlisting}
    \end{itemize}
    {\bf Замечание}: Не путайте ссылки и указатели.
\end{frame}

\begin{frame}[fragile]{Передача параметров по ссылке}
    Передадим наши параметры по ссылке:
    \begin{lstlisting}
void swap(int & a, int & b) {
    int t = a;
    a = b;
    b = t;
}

int main() {
    int k = 10, m = 20;
    swap(k, m);
    std::cout << k << ' ' << m << std::endl; // 20 10
    return 0;
}
    \end{lstlisting}
    Теперь {\tt swap} меняет местами переменные {\tt k} и {\tt m}.
\end{frame}

\begin{frame}[fragile]{Замечания}
    \begin{itemize}
        \item Пишите реализацию элементов из стандартной библиотеки только в образовательных целях.
        \item Если что-то есть в стандартной библиотеке~--- используйте.
        \item Скорее всего реализация стандартной библиотеки эффективнее.
        \item Например, в стандартный алгоритм \code{std::swap} реализован в заголовочном файле \code{<utility>}.
        \item Всегда есть но\dots
    \end{itemize}
\end{frame}

\begin{frame}[fragile]{Указатели}
    \begin{itemize}
        \item Указатель — это переменная, хранящая адрес некоторой 
            ячейки памяти.
        \item Указатели являются типизированными.
\begin{lstlisting}
int i = 3;
// указатель на переменную типа int
int * p = nullptr;
\end{lstlisting}
        \item Нулевому указателю (которому присвоено значение \code{0} или \code{nullptr}) не~соответствует никакая ячейка памяти.
        \item Оператор взятия адреса переменной \code{\&}.
        \item Оператор разыменования \code{*}.
            \begin{lstlisting}
p  = &i; // указатель p указывает на переменную i
*p = 10; // изменяется ячейка по адресу p, т.е. i
            \end{lstlisting}
    \end{itemize}
\end{frame}

\begin{frame}[fragile]{Еще раз о времени жизни переменных}
    \begin{itemize}
        \item Переменные по времени жизни можно разделить на 2 типа: {\bf локальные} и {\bf глобальные}.
        \item Глобальные существуют все время исполнения программы.
        \item Локальные существуют в рамках определенной области~(scope).
        \item Символы \code{\{} и \code{\}} соотвествуют началу и концу области видимости.
        \item В реальности все чуть сложнее\dots
    \end{itemize}
\end{frame}

\begin{frame}[fragile]{Время жизни переменной}
    Следует следить за временем жизни переменных.
    \begin{lstlisting}
int * foo() {
    int a = 10;
    return &a;
}

int & bar() {
    int b = 20;
    return b;
}
    \end{lstlisting}
    \begin{lstlisting}
int * p = foo(); // bad pointer
int & l = bar(); // bad reference
    \end{lstlisting}
\end{frame}

\begin{frame}[fragile]{Массивы}
    \begin{itemize}
        \item {\bf Массив}~--- это набор однотипных элементов, расположенных в памяти
            друг за другом, доступ к которым осуществляется по индексу.
        \item Массиву, как правило, соответствует непрерывный участок памяти.
        \item Непрерывность участка дает преимущество~--- мы более эффективно используем кеш процессора.
    \end{itemize}
\end{frame}

\begin{frame}[fragile]{Статические массивы}
    \begin{itemize}
        \item \langcpp позволяет определять массивы на стеке.
        \item В стиле \langc:
            \begin{lstlisting}
// массив 1 2 3 4 5 0 0 0 0 0
int m[10] = {1, 2, 3, 4, 5}; 
            \end{lstlisting}
        \item Аналог в стиле \langcpp:
            \begin{lstlisting}
#include <array>

std::array<int, 10> m = {1, 2, 3, 4, 5}; 
            \end{lstlisting}
    \end{itemize}
\end{frame}

\begin{frame}[fragile]{Статические массивы}
    \begin{itemize}
        \item Индексация массива начинается с \code{0}, последний элемент
            массива длины {\tt n} имеет индекс {\tt n - 1}.
            \begin{lstlisting}
std::array<int, 10> m = {1, 2, 3, 4, 5}; 

for (int i = 0; i < m.size(); ++i) {
    std::cout << m[i] << ' '; 
}
            \end{lstlisting}
        \item Или range based for:
            \begin{lstlisting}
for (int x : m) {
    // x - ссылка на элемент из m
    std::cout << x << ' '; 
}
            \end{lstlisting}
    \end{itemize}
\end{frame}

\begin{frame}[fragile]{Статические массивы}
    \begin{itemize}
        \item \langcpp умеет выводить типы самостоятельно. Например:
            \begin{lstlisting}
for (const auto & i : m) {
    std::cout << i << ' '; 
}
            \end{lstlisting}
        \item Важно, что ссылка константная. 
            Ведь мы не собираемся изменять содержимое нашего контейнера. 
    \end{itemize}
\end{frame}

\begin{frame}[fragile]{Динамические массивы}
    \begin{itemize}
        \item \langcpp позволяет определять массивы, 
            размер которых не известен на момент компиляции.
        \item В стиле \langcpp:
            \begin{lstlisting}
#include <vector>

std::vector<int> m = {1, 2, 3, 4, 5}; 
            \end{lstlisting}
        \item В стиле \langc обсудим подробнее позже.
    \end{itemize}
\end{frame}

\begin{frame}[fragile]{Динамические массивы}
    \begin{itemize}
    \item Аналогичным образом можем напечать наш динамический массив:
        \begin{lstlisting}
for (const auto & i : m) {
    std::cout << i << ' '; 
}
        \end{lstlisting}
    \item Или так:
        \begin{lstlisting}
for (auto i = 0; i < m.size(); ++i) {
    std::cout << m[i] << ' '; 
}
        \end{lstlisting}
    \end{itemize}
\end{frame}

\begin{frame}[fragile]{Почему следует использовать С++ массивы}
    \begin{itemize}
        \item Обеспечивают большую безопасность.
        \item Достаточно унифицированы (имеют схожие интерфейсы).
        \item ``Дружат’’ с алгоритмами стандартной библиотеки.
        \item Решают большинство ваших задач (не требуется изобретать свой велосипед).
    \end{itemize}
\end{frame}

\begin{frame}{Немного об STL}
    \begin{itemize}
        \item STL = Standard Template Library.
        \item STL является частью стандартной библиотеки \langcpp, описана в стандарте, но не упоминается там явно.
        \item Авторы: Александр Степанов, Дэвид Муссер и Менг Ли.
        \item Основана на разработках для языка Ада.
    \end{itemize}
    Замечание: большинство подходов к разработке библиотек общего назначения и методик обобщенного программирования в целом, 
        используемых в \langcpp, детально описаны в книге: А. Степанов ``Начала программирования’’.
\end{frame}

\begin{frame}{Составляющие STL}
    Можно выделить следующие основные составляющие:
    \begin{itemize}
        \item Контейнеры~--- способ хранения объектов в памяти.
        \item Итераторы~--- унифицированный доступ к элементам контейнера.
        \item Адаптеры~--- обёртки над контейнерами для более удобного использования.
        \item Алгоритмы~--- работа с содержимым контейнеров.
        \item Функциональные объекты, функторы (обобщение функций).
    \end{itemize}
\end{frame}

\begin{frame}{Чего нет в стандартной библиотеке}{}
    \begin{itemize}
        \item Cложных структур данных.
        \item Cложных алгоритмов.
        \item Работы с графикой/звуком.
        \item \dots
    \end{itemize}
\end{frame}

\begin{frame}[fragile]{Лямбда-выражения}
    \begin{lstlisting}
auto op = [](int x, int y) { return x / y; };

// С++14
auto op = [](auto x, auto y) { return x / y; };

// то же, но с указанием типа возвращаемого значения
auto op = [](int x, int y) -> int { return x / y; };
    \end{lstlisting}
\end{frame}

\begin{frame}[fragile]{Лямбда-выражения}    
    Можно захватывать \emph{локальные} переменные.
    \begin{lstlisting}
int total = 0;

// захват по ссылке
auto f1 = [&total](int x) { total += x; };

// захват по значению
auto f2 = [total](int & x) { x -= total ; };

// все по ссылке
auto f3 = [&] { return total; }

// все по значению
auto f3 = [=] { return total; }
    \end{lstlisting}
\end{frame}

\begin{frame}[fragile]{Операции с массивами}
    Сегодня рассмотрим только 2 контейнера:~\code{std::array} и \code{std::vector}, остальные рассмотрим позже.\\
    Что можем поделать с нашими массивами? К примеру:
    \begin{itemize}
        \item Скопировать.
        \item Отсортировать.
        \item Применить некоторый функтор к элементам.
        \item Отфильтровать (удалить элементы).
        \item Применить алгоритмы из стандартной библиотеки.
        \item \dots
    \end{itemize}
\end{frame}

\begin{frame}[fragile]{Операции с массивами}
    Как нам что-то сделать с содержимым массива?
    \begin{itemize}
        \item Написать цикл.
        \item Или вызвать метод контейнера.
        \item Или вызвать функцию из стандатной библотеки.
    \end{itemize}
\end{frame}

\begin{frame}[fragile]{Пример: сортировка массива}
    \begin{itemize}
        \item Объявим и заполним наш массив.
            \begin{lstlisting}
std::vector<int> m = {5, 2, 1, 3, 4};
            \end{lstlisting}
        \item Вариант 1. Пишем сортировку самостоятельно. Например, \enquote{пузырек}:
            \begin{lstlisting}
for (auto i = 0; i < m.size(); ++i) {
    for (auto j = 0; j < m.size() - i - 1; j++) {
        if (m[j] > m[j + 1]) {
            std::swap(m[j], m[j + 1]);
        }
    }
}
            \end{lstlisting}
        \item Вариант 2. Используем алгоритм \code{std::sort}:
            \begin{lstlisting}
std::sort(m.begin(), m.end());
            \end{lstlisting}
    \end{itemize}
\end{frame}

\begin{frame}[fragile]{Пример: копирование массива}
    \begin{itemize}
        \item Объявим и заполним наш массив, создадим контейнер для копии.
            \begin{lstlisting}
std::vector<int> m = {5, 2, 1, 3, 4};
std::vector<int> m_copy;
            \end{lstlisting}
        \item Вариант 1. Пишем самостоятельно:
            \begin{lstlisting}
for (const auto & i : m) {
    m_copy.push_back(i);
}
            \end{lstlisting}
        \item Вариант 2. Используем алгоритм \code{std::copy}:
            \begin{lstlisting}
std::copy(m.cbegin(), m.cend(), m_copy.end());
            \end{lstlisting}
        \item Вариант 3. Используем конструктор:
            \begin{lstlisting}
std::vector<int> m_copy(m.cbegin(), m.cend());
            \end{lstlisting}
    \end{itemize}
\end{frame}

\begin{frame}[fragile]{Пример: применение функции к массиву}
    \begin{itemize}
        \item Объявим и заполним наш массив-строку.
            \begin{lstlisting}
std::string s("hello");
            \end{lstlisting}
        \item Вариант 1. Пишем самостоятельно:
            \begin{lstlisting}
for (auto & c : s) {
    c = std::toupper(c);
}
            \end{lstlisting}
        \item Вариант 2. Используем \code{std::transform}:
            \begin{lstlisting}
const auto f = [](char c) { return std::toupper(c); }
std::transform(s.begin(), s.end(), s.begin(), f);
            \end{lstlisting}
        \item Вариант 3. Используем другой алгоритм из STL:
            \begin{lstlisting}
std::for_each(s.begin(), s.end(),
              [](char c) { std::toupper(c); });
            \end{lstlisting}
    \end{itemize}
\end{frame}

\begin{frame}[fragile]{Пример: удаление элементов из массива}
    \begin{itemize}
        \item Объявим и заполним наш массив.
            \begin{lstlisting}
std::vector<int> m(10);
std::iota(m.begin(), m.end(), 0);
            \end{lstlisting}
        \item Вариант 1. Пишем самостоятельно:
            \begin{lstlisting}
for (auto it = m.begin(); it != m.end();) {
    it = (*it % 2 == 0)
        ? it = m.erase(it)
        : std::next(it);
}
            \end{lstlisting}
    \end{itemize}
\end{frame}

\begin{frame}[fragile]{Пример: удаление элементов из массива}
    \begin{itemize}
        \item Вариант 2. Используем метод:
            \begin{lstlisting}
bool is_odd(int i) {
    return i % 2 == 1;
}

m.erase(
    std::remove_if(v.begin(), v.end(), is_odd),
    v.end());
            \end{lstlisting}
    \end{itemize}
\end{frame}

\begin{frame}[fragile]{Пример: удаление элементов из массива}
    \begin{itemize}
        \item Вариант 3. Используем \code{std::erase}:
            \begin{lstlisting}
bool is_odd(int i) {
    return i % 2 == 1;
}

std::erase_if(m, is_odd); // since C++20
            \end{lstlisting}
    \end{itemize}
\end{frame}

\begin{frame}[fragile]{Пример: удаление элементов из массива}
    \begin{itemize}
        \item Объявим и заполним наш массив-строку.
            \begin{lstlisting}
std::string s("hello");
            \end{lstlisting}
        \item Вариант 1. Вручную:
            \begin{lstlisting}
std::string s_copy = "";
for (auto & c : s) {
    if (c != 'l') {
        s_copy += c;
    }
}
            \end{lstlisting}
        \item Вариант 2. Используем метод контейнера:
            \begin{lstlisting}
s.erase('l');
            \end{lstlisting}
        \item Вариант 3. Алгоритм \code{std::remove\_copy}:
            \begin{lstlisting}
std::remove_copy(str.begin(), str.end(),
                 str.begin(), 'l');
            \end{lstlisting}
    \end{itemize}
\end{frame}

\begin{frame}[fragile]{Пример: подсчет элементов, удовлевторяющих предикату}
    \begin{itemize}
        \item Объявим и заполним наш массив.
            \begin{lstlisting}
std::vector<int> m(10);
std::iota(m.begin(), m.end(), 0);
            \end{lstlisting}
        \item Вариант 1. Вручную:
            \begin{lstlisting}
auto counter = 0u;
for (const auto c : m) {
    if (c % 2 == 0) {
        ++counter;
    }
}
            \end{lstlisting}
        \item Вариант 2. Используем алгоритм:
            \begin{lstlisting}
std::count_if(m.cbegin(), m.cend(),
              [](const auto & i) { return i % 2 == 0; });
            \end{lstlisting}
    \end{itemize}
\end{frame}

\begin{frame}[fragile]{Ranges}
    \begin{itemize}
        \item Упрощают работу с алгоритмами и контейнерами.
        \item Появились в \langcpp[20]. Заголовочный файл \code{<ranges>}.
        \item Не все алгоритмы, адаптеры, \dots были стандартизованы.
        \item Остальная часть широкоиспользуемых вещей доступна в библиотеке RangesV3.
        \item Почти все из RangesV3 скорее всего попадет в \langcpp[23].
    \end{itemize}
\end{frame}

\begin{frame}[fragile]{Ranges примеры}
    \begin{itemize}
        \item Объявим и заполним наш массив.
        \begin{lstlisting}
std::vector<int> m(10);
std::iota(m.begin(), m.end(), 0);
            \end{lstlisting}
        \item Определим наш предикат:
            \begin{lstlisting}
 const auto even = [](int i) { return i % 2 == 0; };
            \end{lstlisting}
        \item Напечатаем наш массив:
            \begin{lstlisting}
for (auto & i : m | std::views::filter(even)) {
    std::cout << i << ' ';
}
            \end{lstlisting}
        \item Другой вариант:
            \begin{lstlisting}
for (auto & i : std::views::filter(m, even)) {
    std::cout << i << ' ';
}
            \end{lstlisting}
    \end{itemize}
\end{frame}

\begin{frame}[fragile]{Ranges примеры}
    \begin{itemize}
        \item Дополнительно определим функтор:
            \begin{lstlisting}
 const auto square = [](int i) { return i * i; };
            \end{lstlisting}
        \item Выполним композицию \code{even} и \code{square}:
            \begin{lstlisting}
for (auto & i : m | std::views::filter(even)
                  | std::views::transform(square)) {
    std::cout << i << ' ';
}
            \end{lstlisting}
    \end{itemize}
\end{frame}

\begin{frame}[fragile]{Ranges примеры}
    \begin{itemize}
        \item Еще раз рассмотрим пример с подсчетом четных чисел:
            \begin{lstlisting}
const auto even = [](int i) { return i % 2 == 0; };
const auto counter = std::ranges::count_if(m, even);
            \end{lstlisting}
        \item Копирование массива:
            \begin{lstlisting}
const auto m_copy = std::ranges::copy(
    std::views::filter(m, even),
    std::back_inserter(m_copy));
            \end{lstlisting}
    \end{itemize}
\end{frame}

\end{document}
