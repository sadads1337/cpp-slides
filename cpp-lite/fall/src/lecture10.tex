\documentclass{beamer}
\usepackage{csc}
\title{Лекция 10. Многопоточное программирование.}

\date{
   \textbf{ИТМО JB}\\
   9 ноября 2021 \\
   Санкт-Петербург
}

\newcommand{\specialcell}[2][c]{%
  \begin{tabular}[#1]{@{}c@{}}#2\end{tabular}}

\begin{document}
\begin{frame} 
  \titlepage
\end{frame}

\begin{frame}[fragile]{Многопоточное исполнение}
    Все современные программы многопоточные. \\
    Какие проблемы решают разработчики?
    \begin{itemize}
        \item Deadlock'и.\\
        \begin{tabular}{ | l | l | l | }
            \hline
            Шаг & Процесс 1 & Процесс 2 \\ \hline
            0 & \specialcell{Хочет захватить A и B,\\ начинает с A} & \specialcell{Хочет захватить A и B,\\ начинает с B} \\ \hline
            1 & Захватывает ресурс A & Захватывает ресурс B \\ \hline
            2 & \specialcell{Ожидает освобождения\\ ресурса B} & \specialcell{Ожидает освобождения\\ ресурса A} \\ \hline
        \end{tabular}
        \item Race condition'ы.
    \end{itemize}
\end{frame}

\begin{frame}[fragile]{Многопоточное исполнение}
    Какие абстракции у нас есть?
    \begin{itemize}
        \item \code{std::thread}.
        \item \code{std::future} и \code{std::promise}.
        \item \code{std::atomic} и CAS.
        \item \code{co\_await}, \code{co\_yield} и \code{co\_return} (\langcpp[20]).
    \end{itemize}
    Какие библиотеки у нас есть?
    \begin{itemize}
        \item pthread.
        \item Boost.
        \item OpenMP и MPI.
        \item OpenCL и CUDA.
        \item \dots
    \end{itemize}
\end{frame}

\begin{frame}[fragile]{Синхронизация}
    Какие инструменты синхронизации есть?
    \begin{itemize}
        \item \code{std::mutex}, \code{std::recursive\_mutex}, \dots
        \item \code{std::lock\_guard}, \code{std::scoped\_lock}, \code{std::unique\_lock}.
        \item \code{std::counting\_semaphore}, \code{std::binary\_semaphore} (\langcpp[20]).
        \item \code{std::conditional\_variable}, \dots
        \item \code{std::memory\_ordering}.
    \end{itemize}
    С помощью этих инструментов мы можем строить более сложные примитивы.
\end{frame}

\begin{frame}[fragile]{Просто кот}
    \begin{center}
        \includegraphics[height=50mm]{cat.jpg}
    \end{center}
\end{frame}

\begin{frame}[fragile]{Асинхронное выполнение}
Пусть мы хотим вычислить \texttt{doAsyncWork} асинхронно.
\begin{lstlisting}
int doAsyncWork(); // many work here
\end{lstlisting}

В \langcpp есть несколько способов выполнения задач асинхронно:
\begin{itemize}
    \item создать поток вручную \code{std::thread},
    \item использование \code{std::async},
    \item использовать корутины.
\end{itemize}
\end{frame}

\begin{frame}[fragile]{Асинхронное выполнение - вариант 1}

\begin{lstlisting}
int doAsyncWork(); // many work here
\end{lstlisting}

\begin{lstlisting}
#include <thread>

int main()
{
    std::thread t(doAsyncWork);
    ...
    t.join();
}
\end{lstlisting}
\end{frame}

\begin{frame}[fragile]{std::thread}
    \begin{itemize}
        \item Сразу же начинает вычислять переданную функцию.

        \item Игнорирует возвращаемое значение функции.
    \end{itemize}
\begin{lstlisting}
int main()
{
    int res = 0; // here we store the result
    std::thread t([&res](){res = doAsyncWork();});
    ...
    t.join();
    // res is ready here.
    ...
}
\end{lstlisting}
\end{frame}

\begin{frame}[fragile]{std::thread}       
    \begin{itemize}
        \item Метод \code{join()} позволяет заблокировать текущий поток,\\ пока
            выполнение потока не завершится.
        \item Метод \code{detach()} позволяет отключить поток от объекта,\\
            т.е. разорвать связь между объектом и потоком.
        \item При вызове деструктора подключаемого потока программа
            завершается, т.е. необходимо вызвать \code{join} или \code{detach}.
        \item Исключения не могут покидать пределы потока.
        \item Метод \code{native\_handle()} возвращает дескриптор потока.
    \end{itemize}
\end{frame}

\begin{frame}[fragile]{Асинхронное выполнение - вариант 2}

\begin{lstlisting}
int doAsyncWork(); // many work here
\end{lstlisting}

\begin{lstlisting}
#include <future>

int main()
{
    std::future<int> fut = std::async(doAsyncWork);
    ...
    int res = fut.get(); // or wait()
}
\end{lstlisting}

\code{std::async} может (зависит от планировщика) отложить выполнение задачи до вызова \code{get} или \code{wait}.
\end{frame}

\begin{frame}[fragile]{std::async}
    \begin{itemize}
        \item Имеет две стратегии выполнения: асинхронное выполнение и
            отложенное (синхронное/ленивое) выполнение.
                \begin{enumerate}
                    \item \code{std::launch::async}
                    \item \code{std::launch::deferred}
                \end{enumerate}
        \item По умолчанию имеет стратегию:\\
             \code{std::launch::async | std::launch::deferred}
\end{itemize}
\begin{lstlisting}
int main()
{
    std::future<int> fut = std::async(
        std::launch::async,
        doAsyncWork);
    ...
    int res = fut.get();
}
\end{lstlisting}
\end{frame}

\begin{frame}[fragile]{std::async}
\begin{itemize}
        \item Отложенная задача может никогда не выполниться,\\
            если не будет вызвано \code{get} или \code{wait}.
        \item Возвращает \code{std::future<T>}, который\\ позволяет
            получить возвращаемое значение.
        \item Позволяет обрабатывать исключения.
    \end{itemize}
\end{frame}

\begin{frame}[fragile]{Асинхронное выполнение - вариант 3}
Функция является корутиной, если выполнено одно из следующих условий:
\begin{itemize}
    \item Использует оператор \code{co\_await} для приостановки исполнения
    \begin{lstlisting}
task<> tcp_echo_server() {
  char data[1024];
  while (true) {
    size_t n = co_await 
        socket.async_read_some(buffer(data));
    co_await async_write(socket, buffer(data, n));
  }
}
    \end{lstlisting}
\end{itemize}
\end{frame}

\begin{frame}[fragile]{Асинхронное выполнение - вариант 3}
Функция является корутиной, если выполнено одно из следующих условий:
\begin{itemize}
    \item Использует ключевое слово \code{co\_return} для завершения исполнения и возврата результата.
    \begin{lstlisting}
lazy<int> f() { co_return 7; }
    \end{lstlisting}
    \item Использует ключевое слово \code{co\_yield} для приостановки исполнения и возврата результата.
    \begin{lstlisting}
generator<int> iota(int n = 0) {
  while(true)
    co_yield n++;
}
    \end{lstlisting}
\end{itemize}
\end{frame}

\begin{frame}[fragile]{Асинхронное выполнение - вариант 3}
Пример с использованием библиотеки \code{cppcoro}:
    \begin{lstlisting}
#include <cppcoro/generator.hpp>
#include <iostream>

cppcoro::generator<int> iota(int n = 0) {
  while(true)
    co_yield n++;
}

void usage()
{
  for (auto i : iota(1000))
  {
    std::cout << i << std::endl;
  }
}
    \end{lstlisting}
\end{frame}

\begin{frame}[fragile]{Синхронизация: mutex}
    \begin{lstlisting}
double shared = 0; std::mutex mtx;

void compute(int begin, int end)  {
    for (int i = begin; i != end; ++i) {
        double current = someFunction(i);        
        // critical section
        std::scoped_lock lock(mtx);
        shared += current;        
    }
}

int main() {
  std::thread th1 (compute, 0,   100);
  std::thread th2 (compute, 100, 200);
  th1.join(); th2.join();
  std::cout << shared << std::endl;
}
\end{lstlisting}
\end{frame}

\begin{frame}[fragile]{Синхронизация: semaphore}
    \begin{lstlisting}
std::binary_semaphore mainToThread(0), threadToMain(0);
void worker() {
	mainToThread.acquire(); // wait
	std::cout << "[thread] Got the signal\n";
	using namespace std::literals;
	std::this_thread::sleep_for(3s);
	std::cout << "[thread] Send the signal\n";
	threadToMain.release(); // send
}
int main() {
	std::jthread thrWorker(worker);
	std::cout << "[main] Send the signal\n";
	mainToThread.release(); // send
	threadToMain.acquire(); // wait
	std::cout << "[main] Got the signal\n";
}
\end{lstlisting}
\end{frame}

\begin{frame}[fragile]{Синхронизация: semaphore}
    \begin{lstlisting}
[main] Send the signal
[thread] Got the signal
[thread] Send the signal
[main] Got the signal
    \end{lstlisting}
\end{frame}

\begin{frame}[fragile]{std::atomic}
    \begin{itemize}
        \item Шаблон \code{std::atomic} позволяет определить переменную,
            операции с которой будут атомарны.
        \item Определён только для целочисленных встроенных типов и указателей.
    \end{itemize}

\begin{lstlisting}
template<class T>
struct shared_ptr_data {
    void addref() {
        ++counter; // atomic increment
    }
    
    T * ptr;
    std::atomic<size_t> counter;
};
\end{lstlisting}
\end{frame}

\begin{frame}[fragile]{Синхронизация: lock-free}
    \begin{lstlisting}
template<class T>
struct Node {
    T data;
    Node * next;
    Node(const T& data)
    : data(data)
    , next(nullptr) {}
};

template<class T>
class Stack {
    std::atomic<node<T>*> head;
public:
    void push(const T& data);
};
\end{lstlisting}
\end{frame}

\begin{frame}[fragile]{Синхронизация: lock-free}
    \begin{lstlisting}
template<class T>
void Stack<T>::push(const T& data)
{
    Noe<T>* new_node = new node<T>(data);
    new_node->next = head.load(std::memory_order_relaxed);
    while(!std::atomic_compare_exchange_weak_explicit(
        &head,
        &new_node->next,
        new_node,
        std::memory_order_release,
        std::memory_order_relaxed))
    ; // the body of the loop is empty
}
    \end{lstlisting}
\end{frame}

\begin{frame}[fragile]{Синхронизация: lock-free}
    \begin{lstlisting}
int main() {
    Stack<int> s;
    s.push(1);
    s.push(2);
    s.push(3);
}
    \end{lstlisting}
\end{frame}

\begin{frame}{Общие советы и замечания}
    \begin{itemize}
        \item При использовании \code{std::thread} следите за тем,\\ чтобы исключения не покидали функцию потока.
        \item Предпочитайте \code{std::async} прямому созданию потоков*.
        \item Критические секции должны быть минимальными.
        \item Используйте \code{std::atomic} вместо мьютекса,\\
            когда синхронизация нужна только для одной\\ целочисленной
            переменной.
        \item Помните про thread-санитайзер.
        \item \code{volatile}~--- это не про многопоточность.
    \end{itemize}
    {\bf*Полезные замечания:}\\
        \href{https://youtu.be/dig9jrcCmLU}{Сергей Видюк, Нестандартный future/promise}
\end{frame}

\end{document}

 
