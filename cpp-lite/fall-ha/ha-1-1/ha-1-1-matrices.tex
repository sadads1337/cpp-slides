% Home assignment "Manipulation with matrices"

% TODO: добавить замечание, что данная задача не требует использования/знания 
% классов, структур, исключений. 
% Достаточно уметь пользоваться двухмерными массивами (выделять, освобождать, 
% итерироваться по ним), потоками ввода/вывода, строками, функциями.
% TODO: наверно стоит добавить требование проверки ошибок чтения из файла
% (на уровне файл существует/не существует, без разбора возможных ошибок
% внутреннего формата файла).

\documentclass[a4paper,10pt]{article}

% Encoding support.
\usepackage{cmap}  % makes pdf files generated with pdflatex scannable and searchable
\usepackage{ucs}
\usepackage[utf8x]{inputenc}
\usepackage[T2A]{fontenc}
\usepackage[russian,english]{babel}

\usepackage{hyperref}

% For images
\usepackage{graphicx}

\usepackage{amsmath, amsthm, amssymb}

% Spaces after commas.
\frenchspacing
% Minimal carrying number of characters,
\righthyphenmin=2

% From K.V.Voroncov Latex in samples, 2005.
\textheight=24cm   % text height
\textwidth=16cm    % text width.
\oddsidemargin=0pt % left side indention
\topmargin=-1.5cm  % top side indention.
\parindent=24pt    % paragraph indent
\parskip=0pt       % distance between paragraphs.
\tolerance=2000
\flushbottom       % page height aligning
\hoffset=0cm
%\pagestyle{empty}  % without numeration

% geometry
\usepackage[a4paper,top=15mm]{geometry}

% Indenting first paragraph.
\usepackage{indentfirst}

%\usepackage{setspace}
%\linespread{1.5}

\usepackage{enumitem}
%\usepackage{datetime}

% Listings
\usepackage{xcolor}
\usepackage{listings}
\usepackage{fancyvrb}

% Auto size brackets in math equations
\usepackage{nath}
\delimgrowth=1

% To remove vertical space after title
\usepackage{titling}

% For directory listings
\usepackage{dirtree}

\begin{document}
\selectlanguage{russian}

\lstset{
  basicstyle=\ttfamily,
  columns=fullflexible
}

\title{\includegraphics[height=15mm]{../mse-logo}\\[1em]
Домашнее задание \textnumero 1 \\ Операции над матрицами}
\preauthor{}
\author{}
\postauthor{}
\date{14 сентября 2021}

\maketitle

\paragraph{Постановка задачи.}
Разработайте утилиту командной строки для сложения и умножения матриц.

Программа должна принимать в качестве аргументов командной строки список файлов с матрицами, 
разделённых командами-операциями, которые необходимо выполнить над матрицами.

Пример запуска в терминале Unix/Linux:

\begin{lstlisting}[language=bash, frame=single]
$ ./matrices mat1.txt --add mat2.txt --mult mat3.txt --mult mat4.txt
\end{lstlisting}

В терминале Windows:

\begin{lstlisting}[language=command.com, frame=single]
C:\> matrices.exe mat1.txt --add mat2.txt --mult mat3.txt --mult mat4.txt
\end{lstlisting}

Необходимо поддержать две операции:
\begin{enumerate}
  \item[1)] \lstinline[language=bash]`--add` (начинается с двух знаков минус)~--- сложение матриц,
  \item[2)] \lstinline[language=bash]`--mult`~--- умножение матриц.
\end{enumerate}

Операции необходимо выполнить в том порядке, в котором они указаны.
Например, в указанном выше примере необходимо сначала сложить матрицу из \texttt{mat1.txt} c матрицей из \texttt{mat2.txt},
затем умножить результат на матрицу из \texttt{mat3.txt},
затем умножить результат на матрицу из \texttt{mat4.txt}:
$$
  (((\mathrm{mat}_1 + \mathrm{mat}_2) \cdot \mathrm{mat}_3) \cdot \mathrm{mat}_4).
$$

Каждая матрица задаётся в отдельном текстовом файле.
На первой строке файла указаны два натуральных числа $N$ и $M$~---
соответственно количество строк и столбцов в матрице.
Далее следует $N$ строк по $M$ чисел с плавающей точкой.
Например:
\begin{Verbatim}[frame=single]
2 5
7.11 3.12 8.13 2.14 0.15
3.21 7.22 3.23 2.24 9.25
\end{Verbatim}

Результирующую матрицу необходимо вывести на экран (в терминал) в том же формате, в котором заданы входные матрицы.
Не допускается выводить какую-либо дополнительную или отладочную информацию.

\paragraph{Обработка ошибок.}
Данная задача не требует обязательного использования/знания исключений.
В случае, если аргументы командной строки не соответствуют указанному выше формату, или, если невозможно произвести операцию над матрицами (например, сложение матриц разного размера),
программа должна вывести осмысленное сообщение об ошибке и завершиться с ненулевым кодом возврата (возвращаемым значением из функции {\tt main}).

Необходимо проверять корректность открытия файла (например, приведя файловый поток к логическому типу), и, 
если файл открыть не удалось, выводить сообщение об ошибке и завершать программу с ненулевым кодом возврата.
Если файл с матрицей успешно открылся, можно допустить, что он всегда будет успешно прочитан, 
и что в нём всегда корректно сохранённая матрица в описанном выше формате.

\paragraph{Дополнительные требования.}
Данная задача {\em не требует} использования или знания классов C++, структур.
Достаточно уметь пользоваться двухмерными массивами (создавать, удалять и итерироваться по ним), потоками ввода/вывода, строками и функциями.

Для хранения и обработки матриц необходимо использовать {\tt std::vector} и его методы (и, если потребуется, алгоритмов из стандартной библиотеки).
Постарайтесь не допустить дублирования кода при работе с матрицами, разумно вынести такие операции в отдельные функции.

Разрешается и рекомендуется использование классов STL для хранения остальных сущностей, помимо матриц. 
Например, для удобной работы со строками можно использовать {\tt std::string}.

Разрешается, но не требуется, написание собственного класса матрицы.

\paragraph{Параметры командной строки.}
Для того, чтобы обработать параметры командной строки, нужно использовать следующий синтаксис функции {\tt main}:
\begin{lstlisting}
int main(int argc, char ** argv)
\end{lstlisting}
{\tt argc} — количество параметров командной строки,
а  {\tt argv }~— массив строк в стиле C. Строка с номером
{\tt 0} в этом массиве — это имя самой команды,
строки с номерами от {\tt 1} до {\tt (argc - 1)}~—
это параметры командной строки.

Например, чтобы вывести все параметры командной строки
можно написать следующий код.
\begin{lstlisting}
#include <iostream>

int main(int argc, char ** argv)
{
    for (int i = 1; i < argc; ++i)
    {
        std::cout << i << ": " << argv[i] << std::endl;
    }
    return 0;
}
\end{lstlisting}

\paragraph{Формат сдачи.}
В директории ha1 в репозитории должны быть четыре файла:
{\tt main.cpp, matrices.hpp, matrices.cpp} и {\tt Makefile}
или {\tt CMakeLists.txt}, если вы используете CMake.

Таким образом, ваша директория в Subversion должна выглядеть следующим образом:
\dirtree{%
.1 ha1.
.2 Makefile.
.2 main.cpp.
.2 matrices.cpp.
.2 matrices.hpp.
}

\paragraph{Часто возникающие ошибки.}

\begin{enumerate}
    \item Используйте прилагающийся к заданию smoke test:
    в нём приведены различные варианты запуска решения с ожидаемыми результатами.
    \item Для ввода/вывода необходимо использовать потоковые интерфейсы C++ ({\tt std::cin, std::cout, std::ifstream}).
    Запрещено использование средств C ({\tt fprintf(), fopen()}).
    \item Ошибки необходимо выводить в {\tt std::cerr}.
    \item Используйте разные экземпляры класса {\tt std::ifstream} для чтения разных файлов.
    Используйте то, что {\tt std::ifstream} автоматически закроет файл,
    когда экземпляр класса будет уничтожен, не вызывайте {\tt std::ifstream::close()} когда это не требуется.
    \item Нет необходимости заранее проверять то, что файлы существуют:
    после проверки, но перед повторным открытием файл могут удалить из файловой системы.
    Проверяйте удалось ли открыть файл после попытки его открыть, используя приведение к {\tt bool} или соответсвующий метод класса.
    \item Убедитесь, что директория с решением и лежащие в файлы называются в точности так, как требуется в условии.
    \item В заголовочных файлах необходимо включать только минимально необходимые заголовочные файлы.
    Например, {\tt <iostream>} скорее всего не нужен в {\tt matrices.hpp}, его стоит включить в cpp файле.
    \item Передавайте неизменяемые экземпляры классов по константной ссылке.
    \item Если метод класса не изменяет состояние класса, сделайте его константным.
    \item Если вы передаёте объект по указателю или ссылке и не будете его модифицировать,
    передавайте его явно константным.
    \item Используйте передачу по ссылке вместо передачи по указателю.
    Указатель может быть нулевым; указатель заставляет задуматься о том,
    должна ли функция освобождать память, на которую указывает указатель.
    \item Убедитесь, что при возникновении ошибок вы корректно освобождаете все выделенные ресуры.
    \item Избегайте дублирования кода: вынесите общий код в отдельные функции.
    \item Не стоит использовать {\tt using namespace std;} в заголовочных файлах,
    т.к. не все пользователи заголовочного файла могут хотеть вносить
    содержимое {\tt namespace std} в корневую область видимости.
    \item Приватные члены класса следует именовать с подчеркиванием на конце.
    \item Не забывайте про необходимость вывести размер матрицы при печати матрицы в консоль.
    \item Используйте корректную терминологию, её можно посмотреть, например,
    в Википедии\footnote{\url{https://en.wikipedia.org/wiki/Matrix_(mathematics)}}.
    В английском языке <<матрица>> --- <<matrix>>, <<матрицы>> --- <<matrices>>.
    Индексация в матрицах традиционно сначала по строкам (<<row>>),
    затем по столбцам (<<column>>).
    \item Если вы используете исключения:
    \begin{enumerate}
        \item В качестве типов исключений либо
        используйте исключения из стандартной библиотеки
        ({\tt std::runtime_error, std::invalid_argument}), либо создайте свой класс
        исключения, унаследовав его от {\tt std::exception} или {\tt std::runtime_error}.
        \item Передавайте текст ошибки в исключении (см. {\tt std::exception::what()}) и
    	выводите его там, где ловите исключение.
    	\item Ловите исключения по константной ссылке, если не собираетесь их модифицировать.
    \end{enumerate}
    \item Имена, начинающиеся с подчеркивания и большой буквы, и имена,
    содержащие два последовательных подчеркивания,
    зарезервированы для реализации компилятора и стандартной библиотеки.
    Такие имена нельзя использовать в стражах включения.
    \item Не стоит передавать скалярные типы ({\tt size_t, int, double} и т.п.) по ссылке.
    \item Нельзя использовать {\tt exit()} для обработки ошибок.
    Функция {\tt exit()} никогда не возвращается и выделенные ранее ресурсы никогда корректно не освобождаются
    (например, явно не освобождается память выделенная внутри {\tt std::string}).
    \item Пользуйтесь инвариантами (assert) корректно, они могут помочь отлаживать программу.
    \item Старайтесь выделять замкнутую функциональность в отдельные функции.
    Например, печать матрицы на экран стоит оформить в виде отдельной функции,
    а не в виде вложенных циклов в середине {\tt main()}.
    \item Используйте средства C++ для работы со строками.
    Не используйте функции C для работы со строками.
    Вместо {\tt strcmp} храните строки в {\tt std::string} и сравнивайте с помощью {\tt ==}.
    \item Используйте value initialization\footnote{\url{http://en.cppreference.com/w/cpp/language/value_initialization}}
    для выделения массива, заполненного нулями, вместо явного заполнения нулями в цикле.
    \item При объявлении функции или метода указывайте имена аргументов,
    это делает интерфейс существенно более понятным.
    \item Используйте тип {\tt size_t} для размеров и индексов,
    диапазона значений типа {\tt int} может не хватить.
    \item Используйте {\tt nullptr} вместо {0} или {\tt NULL} для нулевых указателей.
    \item Проверяйте четность числа, смотря на остаток от его деления на два
    ({\tt n \% 2 == 0}), а не на младший бит.
    \item Следуйте Cpp Core Guidelines\footnote{\url{https://isocpp.github.io/CppCoreGuidelines/CppCoreGuidelines}}.
\end{enumerate}

\paragraph{Сроки сдачи.}
Будет три срока, к которому можно будет сдавать домашнее задание:
\begin{itemize}
    \item 23:59 27 сентября (понедельник),
    \item 23:59 4 октября (понедельник),
    \item 23:59 11 октября (понедельник).
\end{itemize}
Если домашнее задание не принимается с первой попытки, его
можно попробовать сдать со следующей попытки.

\end{document}
