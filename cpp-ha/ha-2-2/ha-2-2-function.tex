% Home assignment "lazy_string"

\documentclass[a4paper,10pt]{article}

% Encoding support.
\usepackage{cmap}  % makes pdf files generated with pdflatex scannable and searchable
\usepackage{ucs}
\usepackage[utf8x]{inputenc}
\usepackage[T2A]{fontenc}
\usepackage[russian,english]{babel}

% For images
\usepackage{graphicx}

\usepackage{amsmath, amsthm, amssymb}
\usepackage{hyperref}

% Spaces after commas.
\frenchspacing
% Minimal carrying number of characters,
\righthyphenmin=2

% From K.V.Voroncov Latex in samples, 2005.
\textheight=24cm   % text height
\textwidth=16cm    % text width.
\oddsidemargin=0pt % left side indention
\topmargin=-1.5cm  % top side indention.
\parindent=24pt    % paragraph indent
\parskip=0pt       % distance between paragraphs.
\tolerance=2000
\flushbottom       % page height aligning
\hoffset=0cm
%\pagestyle{empty}  % without numeration

% geometry
\usepackage[a4paper,top=15mm]{geometry}

% Indenting first paragraph.
\usepackage{indentfirst}

%\usepackage{setspace}
%\linespread{1.5}

\usepackage{enumitem}
%\usepackage{datetime}

% Listings
\usepackage{xcolor}
\usepackage{listings}
\usepackage{fancyvrb}

% Auto size brackets in math equations
\usepackage{nath}
\delimgrowth=1

% To remove vertical space after title
\usepackage{titling}

% For directory listings
\usepackage{dirtree}

\begin{document}
\selectlanguage{russian}

\lstset{
  basicstyle=\ttfamily,
  columns=fullflexible
}

\newcommand{\cpp}[1]{{\tt #1}}

\title{\includegraphics[height=15mm]{../CSCCPP}\\[1em]
Домашнее задание \textnumero 5 \\ Класс \cpp{function}}
\preauthor{}
\author{}
\postauthor{}
\date{26 марта 2017}

\maketitle

\paragraph{Постановка задачи.}
Реализуйте функциональность \cpp{std::function} из стандартной библиотеки C++ (стандарт 2011 года).
\begin{lstlisting}[language=c++,frame=single]
#include <functional>
#include <iostream>
 
void print_num(int i)
{
    std::cout << i << '\n';
}
 
struct PrintNum {
    void operator()(int i) const
    {
        std::cout << i << '\n';
    }
};
 
int main()
{
    // store a free function
    std::function<void(int)> f_display = print_num;
    f_display(-9);
 
    // store a call to a function object
    std::function<void(int)> f_display_obj = PrintNum();
    f_display_obj(18);
}
\end{lstlisting}

Ваш объект \cpp{function} должен частично реализовывать функционал стандартного
\cpp{std::function}’a. А именно:
\begin{enumerate}
    \item Должен определяться прототипом функции, т.е. объявление объекта функции может
        выглядеть следующим образом \cpp{function<void(int, double)>}.
    \item Допускает конструирование:
    \begin{enumerate}
        \item от функции (или указателя на нее) с таким же прототипом, какой указан при
        объявлении объекта function,
        \item от функтора (объекта с оператором \cpp{()}),
        \item от \cpp{nullptr},
        \item без параметров.
    \end{enumerate}
    \item Предоставляет операторы:
    \begin{enumerate}
        \item (). Попытка передать параметры в этот оператор, которые не могут быть
преобразованы к типам, принимаемым хранящимися внутри функтором или
функцией, должны приводить к ошибке компиляции (а не времени выполнения).
Оператор должен иметь возможность быть примененным к константному объекту.
Возвращаемый тип должен соответствовать прототипу функции. Попытка вызвать
данный оператор у пустого объекта \cpp{function} должна приводить к генерации
исключения \cpp{fn::bad_function_call}, в противном случае оператор должен приводить
к вызову хранящегося внутри функтора или функции.
        \item Приведения к выражению логического типа. Для проверки на пустоту функции.
    \end{enumerate}
    \item Должны быть функции \cpp{swap} (как \cpp{member}, так и внешняя).
    \item Обеспечивает копирование и перемещение (\cpp{move}).
\end{enumerate}

\paragraph{Общие требования.}
\begin{enumerate}
    \item Предполагается, что реализация использует variadic templates.
    \item При передаче параметров функции должен использоваться perfect
        forwarding (rvalue reference $+$ \cpp{std::forward}).
\item Реализация не должна вводить дополнительных пессимизаций. Однако, если объект
    \cpp{function} создан по функтору, то допускается:
    \begin{enumerate}
        \item неоптимизируемый компилятором косвенный вызов оператора \cpp{()};
        \item выделение динамической памяти для хранения самого функтора.
    \end{enumerate}
    Во всех же остальных случаях следует максимально избегать накладных расходов, которые
    возникают из-за использования \cpp{function}’a, которые компилятор не может
    разрешить в силу недостаточности знаний о ваших намерениях.
    
\item Не забывайте о самодостаточности вашего заголовочного файла. Пользователь должен
иметь возможность включить его в «пустой» cpp-шник и иметь возможность использовать
ваш \cpp{function}. Также, не должно возникать warning’ов, если пользователь ваш
заголовочный файл включил, но ничего из него не использует.
    \item Ваша реализация должна обеспечивать строгую гарантию безопасности исключений.
\end{enumerate}

\paragraph{Подсказки.} 
\begin{enumerate}
    \item Для начала попробуйте реализовать всё, ограничившись функциями от
        одного аргумента, и только потом добавьте возможность работы с
        произвольным набором аргументов.

    \item Для решения данной задачи вам потребуется придумать, как
        хранить внутри объекта \cpp{function} объект произвольного типа (в примере выше
        в объекте \cpp{function<void(int)>} в одном случае хранится указатель на
        функцию, а в другом случае~--- функциональный объект). Подсказка: для этого нужно 
        использовать смесь шаблонов и ООП.

    \item Для корректной реализации вам обязательно нужно просмотреть набор
        шаблонов для работы с типами \url{http://en.cppreference.com/w/cpp/types}.
\end{enumerate}


\paragraph{Формат сдачи.}
Решение необходимо оформить в виде библиотеки, состоящей из одного заголовочного файла.
Заголовочный файл должен называться {\tt fn.hpp}.

Класс \cpp{function} и связанные определения типов должны быть реализованы в
пространстве имен \cpp{fn}. Все необходимые вспомогательные классы и функции
должны быть либо скрыты внутри класса \cpp{function}, либо находиться в
пространстве имён \cpp{fn::details}.

Допускается нахождение в репозитории рядом с файлом {\tt fn.hpp} вспомогательных
файлов, которые вы можете использовать для тестирования вашей библиотеки, а
также {\tt Makefile/CMakeLists.txt}, они не будут участвовать в проверке.

Таким образом, ваша директория в Subversion должна выглядеть примерно следующим образом:
\dirtree{%
.1 ha5.
.2 Makefile.
.2 test.cpp.
.2 fn.hpp.
}

\paragraph{Сроки сдачи.}
Будет три срока, к которым можно будет сдавать домашнее задание:
\begin{itemize}
    \item 8:00 3 апреля 2017 (понедельник),
    \item 8:00 10 апреля 2017 (понедельник), 
    \item 8:00 17 апреля 2017 (понедельник).
\end{itemize}
Если домашнее задание не принимается с первой попытки, его
можно попробовать сдать со следующей попытки.

\end{document}
