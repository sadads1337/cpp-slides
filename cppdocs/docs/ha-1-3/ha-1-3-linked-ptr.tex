% Home assignment "linked_ptr"

\documentclass[a4paper,10pt]{article}

% Encoding support.
\usepackage{cmap}  % makes pdf files generated with pdflatex scannable and searchable
\usepackage{ucs}
\usepackage[utf8x]{inputenc}
\usepackage[T2A]{fontenc}
\usepackage[russian,english]{babel}

% For images
\usepackage{graphicx}

\usepackage{amsmath, amsthm, amssymb}

% Spaces after commas.
\frenchspacing
% Minimal carrying number of characters,
\righthyphenmin=2

% From K.V.Voroncov Latex in samples, 2005.
\textheight=24cm   % text height
\textwidth=16cm    % text width.
\oddsidemargin=0pt % left side indention
\topmargin=-1.5cm  % top side indention.
\parindent=24pt    % paragraph indent
\parskip=0pt       % distance between paragraphs.
\tolerance=2000
\flushbottom       % page height aligning
\hoffset=0cm
%\pagestyle{empty}  % without numeration

% geometry
\usepackage[a4paper,top=15mm]{geometry}

% Indenting first paragraph.
\usepackage{indentfirst}

%\usepackage{setspace}
%\linespread{1.5}

\usepackage{enumitem}
%\usepackage{datetime}

% Listings
\usepackage{xcolor}
\usepackage{listings}
\usepackage{fancyvrb}

% Auto size brackets in math equations
\usepackage{nath}
\delimgrowth=1

% To remove vertical space after title
\usepackage{titling}

% For directory listings
\usepackage{dirtree}

\begin{document}
\selectlanguage{russian}

\lstset{
  basicstyle=\ttfamily,
  columns=fullflexible
}

\newcommand{\cpp}[1]{{\tt #1}}

\title{\includegraphics[height=15mm]{../CSCCPP}\\[1em]
Домашнее задание \textnumero 3 \\ Умный указатель \cpp{linked\_ptr}}
\preauthor{}
\author{}
\postauthor{}
\date{13 ноября 2017}

\maketitle

\paragraph{Постановка задачи.}

Разработайте шаблонный класс \cpp{linked\_ptr<T>}, реализующий RAII обертку над указателем.
Шаблонным параметром является тип, указатель на который моделирует \cpp{linked\_ptr}.

\paragraph{Описание.} \cpp{linked\_ptr}~--- умный указатель (т.\,е. обёртка над обычным указателем), позволяющий совместно использовать хранимый указатель из нескольких экземпляров класса \cpp{linked\_ptr}.
Память по указателю, хранимому в \cpp{linked\_ptr}, будет освобождена только тогда, когда все экземпляры \cpp{linked\_ptr}, хранящие этот указатель, будут уничтожены, либо из них этот указатель будет явно высвобожден.

При копировании умного указателя \cpp{a} в \cpp{b} (или при инициализации \cpp{b(a)}), в \cpp{a} добавляется указатель на \cpp{b}, а в \cpp{b}~--- указатель на \cpp{a}.
Таким образом, все экземпляры класса \cpp{linked\_ptr}, которые хранят один и тот же указатель, оказываются в двусвязном списке~--- отсюда и название данного умного указателя.

По своей функциональности \cpp{linked\_ptr} практически эквивалентен \cpp{std::shared\_ptr}.
По сравнению с последним \cpp{linked\_ptr} обладает меньшим временем инициализации,
т.\,к. не выделяет дополнительных блоков памяти (как \cpp{shared\_ptr} для счетчика ссылок),
а хранит все указатели, работающие с одним объектом, в списке. Впрочем, такая реализация требует от \cpp{linked\_ptr} большего времени на копирование и удаление, нежели \cpp{shared\_ptr}.

\paragraph{Требования.}
\begin{enumerate}
 \item Ваша реализация должна соответствовать описанной выше идее поддержания связанных указателей в общем списке.
 \item В реализации не должно быть выделений динамической памяти, которые можно избежать.
 Таким образом, например, нельзя использовать \cpp{std::list} для хранения элементов в списке.
 \item Связывание и удаление умного указателя из общего списка должно выполняться за $O(1)$.
 \item Реализация должна обеспечивать требование определения типа параметра шаблона к моменту его удаления из деструктора \cpp{linked\_ptr}.
 \item Интерфейс \cpp{linked\_ptr} должен содержать следующие методы, аналогичные методам \cpp{std::shared\_ptr}:
 \begin{enumerate}
  \item Конструкторы \cpp{linked\_ptr<T>}:
  \begin{enumerate}
   \item без параметров;
   \item от \cpp{T *}, константного \cpp{linked\_ptr<T>};
   \item а также от \cpp{U *} и \cpp{linked\_ptr<U>}, если \cpp{U *} неявно приводим к \cpp{T *}.
  \end{enumerate}
  \item Конструкторы от одного аргумента не должны задавать дополнительное приведение типа.
  \item Операторы присваивания от \cpp{linked\_ptr<T>} и \cpp{linked\_ptr<U>}.
  Условия аналогичны конструкторам.
  \item Семейство функций \cpp{reset}:
   \begin{enumerate}
    \item без параметров;
    \item от \cpp{T *};
    \item от \cpp{U *} (см.~выше).
   \end{enumerate}
  \item методы \cpp{swap()}, \cpp{get()}, \cpp{unique()} аналогичные методам из \cpp{std::shared\_ptr}.
  \item Операторы \cpp{->} и разыменования.
  \item Операторы проверки на равенство/неравенство.
  \item Операторы сравнения (например, для использования экземпляров \cpp{linked\_ptr<T>} в качестве элементов \cpp{std::set}).
  \item Безопасное приведение к логическому выражению.
 \end{enumerate}
 \item Реализация должна соответствовать требованиям из Правил сдачи,
 т.\,е., например, в ней не должно быть утечек памяти,
 она должна быть оформлена в соответствии с указанными принципами кодирования и т.\,п.
\end{enumerate}

\paragraph{Дополнительное задание.}
При использовании вашего класса с неким шаблонным параметром \cpp{C} требуйте полностью определённый тип параметра шаблона \cpp{C} только в момент инициализации переменной типа \cpp{linked\_ptr<C>}, но не, например, в момент объявления переменной типа \cpp{linked\_ptr<C>}.

\paragraph{Примечания.}
\begin{itemize}
 \item Т.\,к. интерфейс \cpp{linked\_ptr} является подмножеством \cpp{std::shared\_ptr}, настоятельно рекомендуется использовать интерфейс \cpp{std::shared\_ptr} в качестве образца.
 Явно не специфицированная в данном задании функциональность должна работать также, как в \cpp{std::shared\_ptr}, например,
 см.~как работает \cpp{unique()} на неинициализированном \cpp{std::shared\_ptr}.
\end{itemize}

\paragraph{Формат сдачи.}
Решение необходимо оформить в виде библиотеки, состоящей из одного заголовочного файла.
Заголовочный файл должен называться {\tt linked\_ptr.hpp}.

Класс \cpp{linked\_ptr} должен находиться в пространстве имён \cpp{smart\_ptr}.
Все необходимые вспомогательные классы и функции должны быть либо скрыты внутри класса \cpp{linked\_ptr}, либо находиться в пространстве имён \cpp{smart\_ptr::details}.

Допускается нахождение в репозитории рядом с файлом {\tt linked\_ptr.hpp} вспомогательных файлов, которые вы можете использовать для тестирования вашей библиотеки, а также {\tt Makefile/CMakeLists.txt}, они не будут участвовать в проверке.

Таким образом, ваша директория в Subversion должна выглядеть примерно следующим образом:
\dirtree{%
.1 ha3.
.2 Makefile.
.2 test.cpp.
.2 linked\_ptr.hpp.
}

\paragraph{Сроки сдачи.}
Будет три срока, к которым можно будет сдавать домашнее задание:
\begin{itemize}
    \item 8:00 20 ноября (понедельник),
    \item 8:00 27 ноября (понедельник),
    \item 8:00 04 декабря (понедельник).
\end{itemize}
Если домашнее задание не принимается с первой попытки, его
можно попробовать сдать со следующей попытки.

\end{document}
