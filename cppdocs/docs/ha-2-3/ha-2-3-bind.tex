% Home assignment "bind"

\documentclass[a4paper,10pt]{article}

% Encoding support.
\usepackage{cmap}  % makes pdf files generated with pdflatex scannable and searchable
\usepackage{ucs}
\usepackage[utf8x]{inputenc}
\usepackage[T2A]{fontenc}
\usepackage[russian,english]{babel}

% For images
\usepackage{graphicx}

\usepackage{amsmath, amsthm, amssymb}
\usepackage{hyperref}

% Spaces after commas.
\frenchspacing
% Minimal carrying number of characters,
\righthyphenmin=2

% From K.V.Voroncov Latex in samples, 2005.
\textheight=24cm   % text height
\textwidth=16cm    % text width.
\oddsidemargin=0pt % left side indention
\topmargin=-1.5cm  % top side indention.
\parindent=24pt    % paragraph indent
\parskip=0pt       % distance between paragraphs.
\tolerance=2000
\flushbottom       % page height aligning
\hoffset=0cm
%\pagestyle{empty}  % without numeration

% geometry
\usepackage[a4paper,top=15mm]{geometry}

% Indenting first paragraph.
\usepackage{indentfirst}

%\usepackage{setspace}
%\linespread{1.5}

\usepackage{enumitem}
%\usepackage{datetime}

% Listings
\usepackage{xcolor}
\usepackage{listings}
\usepackage{fancyvrb}

% Auto size brackets in math equations
\usepackage{nath}
\delimgrowth=1

% To remove vertical space after title
\usepackage{titling}

% For directory listings
\usepackage{dirtree}

\begin{document}
\selectlanguage{russian}

\lstset{
  basicstyle=\ttfamily,
  columns=fullflexible
}

\newcommand{\cpp}[1]{{\tt #1}}

\title{\includegraphics[height=15mm]{../CSCCPP}\\[1em]
Домашнее задание \textnumero 6 \\ Функция \cpp{bind}}
\preauthor{}
\author{}
\postauthor{}
\date{22 апреля 2017}

\maketitle

\paragraph{Постановка задачи.}
Реализуйте функциональность \cpp{std::bind} из стандартной библиотеки C++ (стандарт 2011 года).
\begin{lstlisting}[language=c++,frame=single]
#include <iostream>
#include <functional>
 
void f(int n1, int n2)
{
    std::cout << n1 << ' ' << n2 << '\n';
}
 
int g(int n1)
{
    return n1;
}
 
int main()
{
    using namespace std::placeholders;
 
    // demonstrates argument reordering
    auto f1 = std::bind(f, _2, _1);
    f1(1, 2, 1001); // 1 is bound by _1, 2 is bound by _2, 1001 is unused

    auto f2 = std::bind(g, 10);
    int x = f2(); // x = g(10)
}
\end{lstlisting}

Ваша функция \cpp{bind} должна частично реализовывать функционал стандартной
функции \cpp{std::bind}’a. А именно:
\begin{enumerate}
    \item Позволяет зафиксировать (забиндить) обычную свободную функцию (передачу
        функтора или member функции можно не поддерживать).
    \item Можно зафиксировать любые подходящие значения в качестве параметра функции.
    \item Можно зафиксировать плейсхолдер (\verb!_1! и/или \verb!_2!), обозначающие какой аргумент
        при вызове будет подставляться в качестве текущего аргумента функции.
    \item Функция \cpp{bind} (назовем биндер) создает объект, который допускает его вызов по
        оператору \cpp{()}. Переданные параметры вместе с зафиксированными должны приводиться
        к типам, ожидаемым зафиксированной функцией. В противном случае такой вызов не
        должен компилироваться.
    \item Биндер допускает вызов по оператору () с б\'{о}льшим количеством аргументов, чем
        это может предполагать переданная в \cpp{bind} функция с возможно зафиксированными
        аргументами. Лишние аргументы игнорируются. В «общих требованиях» описаны
        ограничения на количество параметров.
    \item Биндер допускает копирование и присваивание. Move биндера допускается, если
        все зафиксированные аргументы допускают \cpp{move}.
    \item Тип, возвращаемый оператором () объекта биндера, должен соответствовать
        возвращаемому типу переданной в bind функции.
\end{enumerate}

\paragraph{Общие требования.}
\begin{enumerate}
    \item Предполагается, что реализация использует variadic templates.
    \item Вводится     ограничение на количество плейсхолдеров. Есть только
        \verb!_1! и \verb!_2!.
    \item Реализация \cpp{bind}’a должна корректно работать как с передачей
        аргументов по значению, так и со ссылками (как константными, так и 
        неконстантными). 
    \item При передаче параметров функции должен использоваться perfect
        forwarding (rvalue reference $+$ \cpp{std::forward}).
    \item Реализация не должна вводить дополнительных пессимизаций. 
        Cледует максимально избегать накладных расходов, которые
        возникают из-за использования \cpp{bind}’a, которые компилятор не
        может  разрешить в силу недостаточности знаний о ваших намерениях.
    \item Функция bind,  должны находиться в \cpp{namespace
        fn}.
        Функция \cpp{bind}, \verb!_1!, \verb!_2! и связанные определения типов должны быть реализованы в пространстве имен \cpp{fn}.
        Все необходимые вспомогательные классы и функции должны находиться в пространстве имён \cpp{fn::details}.
    \item Не забывайте о самодостаточности вашего заголовочного файла. Пользователь
        должен иметь возможность включить его в «пустой» cpp-шник и иметь возможность
        использовать ваш \cpp{bind}. Также, не должно возникать warning’ов, если
        пользователь ваш хидер включил, но ничего из него не использует.
    \item Ваша реализация должна обеспечивать строгую гарантию безопасности
        исключений.
\end{enumerate}

\paragraph{Формат сдачи.}
Решение необходимо оформить в виде библиотеки, состоящей из одного заголовочного файла.
Заголовочный файл должен называться {\tt bind.hpp}.

Допускается нахождение в репозитории рядом с файлом {\tt bind.hpp} вспомогательных
файлов, которые вы можете использовать для тестирования вашей библиотеки, а
также {\tt Makefile/CMakeLists.txt}, они не будут участвовать в проверке.

Таким образом, ваша директория в Subversion должна выглядеть примерно следующим образом:
\dirtree{%
.1 ha6.
.2 Makefile.
.2 test.cpp.
.2 bind.hpp.
}

\paragraph{Сроки сдачи.}
Будет три срока, к которым можно будет сдавать домашнее задание:
\begin{itemize}
    \item 8:00 30 апреля 2017 (воскресенье),
    \item 8:00 7 мая 2017 (воскресенье),
    \item 8:00 14 мая 2017 (воскресенье).
\end{itemize}
Если домашнее задание не принимается с первой попытки, его
можно попробовать сдать со следующей попытки.

\end{document}
